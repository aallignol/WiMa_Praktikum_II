% -*- mode: noweb; ess-noweb-default-code-mode: R-mode; -*-
\documentclass[paper=screen,mathserif]{beamer}\usepackage[]{graphicx}\usepackage[]{color}
%% maxwidth is the original width if it is less than linewidth
%% otherwise use linewidth (to make sure the graphics do not exceed the margin)
\makeatletter
\def\maxwidth{ %
  \ifdim\Gin@nat@width>\linewidth
    \linewidth
  \else
    \Gin@nat@width
  \fi
}
\makeatother

\definecolor{fgcolor}{rgb}{0.345, 0.345, 0.345}
\newcommand{\hlnum}[1]{\textcolor[rgb]{0.686,0.059,0.569}{#1}}%
\newcommand{\hlstr}[1]{\textcolor[rgb]{0.192,0.494,0.8}{#1}}%
\newcommand{\hlcom}[1]{\textcolor[rgb]{0.678,0.584,0.686}{\textit{#1}}}%
\newcommand{\hlopt}[1]{\textcolor[rgb]{0,0,0}{#1}}%
\newcommand{\hlstd}[1]{\textcolor[rgb]{0.345,0.345,0.345}{#1}}%
\newcommand{\hlkwa}[1]{\textcolor[rgb]{0.161,0.373,0.58}{\textbf{#1}}}%
\newcommand{\hlkwb}[1]{\textcolor[rgb]{0.69,0.353,0.396}{#1}}%
\newcommand{\hlkwc}[1]{\textcolor[rgb]{0.333,0.667,0.333}{#1}}%
\newcommand{\hlkwd}[1]{\textcolor[rgb]{0.737,0.353,0.396}{\textbf{#1}}}%

\usepackage{framed}
\makeatletter
\newenvironment{kframe}{%
 \def\at@end@of@kframe{}%
 \ifinner\ifhmode%
  \def\at@end@of@kframe{\end{minipage}}%
  \begin{minipage}{\columnwidth}%
 \fi\fi%
 \def\FrameCommand##1{\hskip\@totalleftmargin \hskip-\fboxsep
 \colorbox{shadecolor}{##1}\hskip-\fboxsep
     % There is no \\@totalrightmargin, so:
     \hskip-\linewidth \hskip-\@totalleftmargin \hskip\columnwidth}%
 \MakeFramed {\advance\hsize-\width
   \@totalleftmargin\z@ \linewidth\hsize
   \@setminipage}}%
 {\par\unskip\endMakeFramed%
 \at@end@of@kframe}
\makeatother

\definecolor{shadecolor}{rgb}{.97, .97, .97}
\definecolor{messagecolor}{rgb}{0, 0, 0}
\definecolor{warningcolor}{rgb}{1, 0, 1}
\definecolor{errorcolor}{rgb}{1, 0, 0}
\newenvironment{knitrout}{}{} % an empty environment to be redefined in TeX

\usepackage{alltt}

\usetheme{CambridgeUS} 
\useinnertheme{circles}
\useoutertheme[footline=authortitle,subsection = false]{miniframes}
\setbeamercolor{palette tertiary}{fg=white, bg=white!42!black}
\setbeamercolor{alerted text}{fg=red!73!black}

%%%%%%
%\usepackage{Sweave}
\usepackage{natbib}     % for references
\usepackage[osf]{sourcesanspro}
\usepackage{sourcecodepro}
\usepackage{booktabs}
\usepackage{eulervm}
%\renewcommand{\ttdefault}{sourcecodepro}
\usepackage{import}
\usepackage{prodint}
\usepackage{bbm}
\usepackage{tabularx}
\usepackage{dcolumn}
\usepackage{color}
\usepackage{booktabs}
\usepackage{graphicx,rotating,epsfig,multirow,multicol,hhline}
\usepackage{amsmath,amsthm,amssymb,amsfonts}

%% \newcommand{\subfloat}[2][need a sub-caption]{\subcaptionbox{#1}{#2}}

\usepackage{listings}
\lstset{
  basicstyle=\tiny\ttfamily, % Standardschrift
  % numbers=left,               % Ort der Zeilennummern
  %numberstyle=\tiny,          % Stil der Zeilennummern
  % stepnumber=2,               % Abstand zwischen den Zeilennummern
  numbersep=5pt,              % Abstand der Nummern zum Text
  tabsize=2,                  % Groesse von Tabs
  extendedchars=true,         %
  breaklines=true,            % Zeilen werden Umgebrochen
  keywordstyle=\color{blue},
  frame=b,         
  stringstyle=\color{white}\ttfamily, % Farbe der String
  showspaces=false,           % Leerzeichen anzeigen ?
  showtabs=false,             % Tabs anzeigen ?
}

%% \usepackage{subcaption}

\newcommand{\ft}[1]{\frametitle{#1}}
\newcommand{\fst}[1]{\framesubtitle{#1}}

\newenvironment{xframe}[1][]
{\begin{frame}[fragile,environment=xframe]
    \frametitle{#1}}
  {\end{frame}}

\title[Data Manipulation]{Data Manipulation}

\author{Arthur Allignol}

\institute[]{\scriptsize{\url{arthur.allignol@uni-ulm.de}}}

\date{}
%%%%%%

\makeatletter
\setbeamertemplate{footline}
{
  \leavevmode%
  \hbox{%
  \begin{beamercolorbox}[wd=.333333\paperwidth,ht=2.25ex,dp=1ex,center]{author in head/foot}%
    \usebeamerfont{author in head/foot}\insertshortauthor%~~\beamer@ifempty{\insertshortinstitute}{}{(\insertshortinstitute)}
  \end{beamercolorbox}%
  \begin{beamercolorbox}[wd=.333333\paperwidth,ht=2.25ex,dp=1ex,center]{title in head/foot}%
    \usebeamerfont{title in head/foot}\insertshorttitle
  \end{beamercolorbox}%
  \begin{beamercolorbox}[wd=.333333\paperwidth,ht=2.25ex,dp=1ex,right]{date in head/foot}%
    \usebeamerfont{date in head/foot}\insertshortdate{}\hspace*{2em}
    \insertframenumber\hspace*{2ex} 
  \end{beamercolorbox}}%
  \vskip0pt%
}
\makeatother


\AtBeginSection[]
{
  \begin{frame}
    \frametitle{Table of Contents}
    \tableofcontents[currentsection]
  \end{frame}
}
\IfFileExists{upquote.sty}{\usepackage{upquote}}{}
\begin{document}

%%%%%% title page
\newcommand{\titlep}{yes}  % for titlepagelayout

{
\renewcommand{\insertframenumber}{}   % no page number on titlepage
\begin{frame}
\addtocounter{framenumber}{-1}
\titlepage
\end{frame}
}



\section{Factors and Dates}

\begin{frame}[fragile]
  \ft{Factors}
  \begin{itemize}
  \item Factor are variables in R that take on a limited number of
    different values
    \begin{itemize}
    \item Categorical variables
    \item Ordinal variables
    \end{itemize}
  \item Factors are useful for statistical modelling as ordinal
    variables should be treated differently than continuous variables
  \item Factors are also useful for statistical report
    generation. Think SAS labels
  \end{itemize}
\end{frame}

\begin{frame}[fragile]
  \ft{Factors}
  \begin{itemize}
  \item Factors are stored internally as numeric values

  \item A corresponding set of characters is used for displaying
  \end{itemize}
\begin{knitrout}\scriptsize
\definecolor{shadecolor}{rgb}{0.969, 0.969, 0.969}\color{fgcolor}\begin{kframe}
\begin{alltt}
\hlstd{aa} \hlkwb{<-} \hlkwd{factor}\hlstd{(}\hlkwd{c}\hlstd{(}\hlstr{"cats"}\hlstd{,} \hlstr{"dogs"}\hlstd{,} \hlstr{"apples"}\hlstd{))}
\hlstd{aa}
\end{alltt}
\begin{verbatim}
## [1] cats   dogs   apples
## Levels: apples cats dogs
\end{verbatim}
\begin{alltt}
\hlkwd{as.integer}\hlstd{(aa)}
\end{alltt}
\begin{verbatim}
## [1] 2 3 1
\end{verbatim}
\end{kframe}
\end{knitrout}
\end{frame}

\begin{frame}[fragile]
  \ft{Factor Creation}
  \begin{itemize}
  \item Factors are created using the {\tt factor} function
  \item The {\tt levels} argument permits to control the order
  \item The {\tt labels} argument is used to change the levels' names
  \item \verb|ordered = TRUE| creates an ordered factor (ordinal
    variable)
  \end{itemize}
\begin{knitrout}\scriptsize
\definecolor{shadecolor}{rgb}{0.969, 0.969, 0.969}\color{fgcolor}\begin{kframe}
\begin{alltt}
\hlkwd{set.seed}\hlstd{(}\hlnum{21324}\hlstd{)}
\hlstd{data} \hlkwb{<-} \hlkwd{sample}\hlstd{(}\hlkwd{c}\hlstd{(}\hlnum{1}\hlstd{,} \hlnum{2}\hlstd{,} \hlnum{3}\hlstd{),} \hlnum{10}\hlstd{,} \hlnum{TRUE}\hlstd{)}
\hlstd{f0} \hlkwb{<-} \hlkwd{factor}\hlstd{(data)}
\hlstd{f1} \hlkwb{<-} \hlkwd{factor}\hlstd{(data,} \hlkwc{levels} \hlstd{=} \hlkwd{c}\hlstd{(}\hlnum{2}\hlstd{,} \hlnum{3}\hlstd{,} \hlnum{1}\hlstd{))}
\hlstd{f2} \hlkwb{<-} \hlkwd{factor}\hlstd{(data,} \hlkwc{labels} \hlstd{=} \hlkwd{c}\hlstd{(}\hlstr{"I"}\hlstd{,} \hlstr{"II"}\hlstd{,} \hlstr{"III"}\hlstd{))}
\hlstd{f3} \hlkwb{<-} \hlkwd{factor}\hlstd{(data,} \hlkwc{levels} \hlstd{=} \hlkwd{c}\hlstd{(}\hlnum{2}\hlstd{,} \hlnum{3}\hlstd{,} \hlnum{1}\hlstd{),}
             \hlkwc{labels} \hlstd{=} \hlkwd{c}\hlstd{(}\hlstr{"II"}\hlstd{,} \hlstr{"III"}\hlstd{,} \hlstr{"I"}\hlstd{))}
\end{alltt}
\end{kframe}
\end{knitrout}
\end{frame}

\begin{frame}[fragile]
\begin{knitrout}\scriptsize
\definecolor{shadecolor}{rgb}{0.969, 0.969, 0.969}\color{fgcolor}\begin{kframe}
\begin{alltt}
\hlkwd{table}\hlstd{(f0)}
\end{alltt}
\begin{verbatim}
## f0
## 1 2 3 
## 4 3 3
\end{verbatim}
\end{kframe}
\end{knitrout}
\begin{knitrout}\scriptsize
\definecolor{shadecolor}{rgb}{0.969, 0.969, 0.969}\color{fgcolor}\begin{kframe}
\begin{alltt}
\hlkwd{table}\hlstd{(f1)}
\end{alltt}
\begin{verbatim}
## f1
## 2 3 1 
## 3 3 4
\end{verbatim}
\end{kframe}
\end{knitrout}
\begin{knitrout}\scriptsize
\definecolor{shadecolor}{rgb}{0.969, 0.969, 0.969}\color{fgcolor}\begin{kframe}
\begin{alltt}
\hlkwd{table}\hlstd{(f2)}
\end{alltt}
\begin{verbatim}
## f2
##   I  II III 
##   4   3   3
\end{verbatim}
\end{kframe}
\end{knitrout}
\begin{knitrout}\scriptsize
\definecolor{shadecolor}{rgb}{0.969, 0.969, 0.969}\color{fgcolor}\begin{kframe}
\begin{alltt}
\hlkwd{table}\hlstd{(f3)}
\end{alltt}
\begin{verbatim}
## f3
##  II III   I 
##   3   3   4
\end{verbatim}
\end{kframe}
\end{knitrout}

\end{frame}

\begin{frame}[fragile]
  \ft{Factors}
  
  \begin{itemize}
  \item The \verb=levels()= function can be used to change the labels
    once a factor has been created
\begin{knitrout}\scriptsize
\definecolor{shadecolor}{rgb}{0.969, 0.969, 0.969}\color{fgcolor}\begin{kframe}
\begin{alltt}
\hlkwd{levels}\hlstd{(f0)} \hlkwb{<-} \hlkwd{c}\hlstd{(}\hlstr{"I"}\hlstd{,} \hlstr{"II"}\hlstd{,} \hlstr{"III"}\hlstd{)}
\hlstd{f0}
\end{alltt}
\begin{verbatim}
##  [1] II  I   III III III I   I   II  I   II 
## Levels: I II III
\end{verbatim}
\end{kframe}
\end{knitrout}
\item The reference level of a factor can be changed using the
  \verb=relevel= function
\begin{knitrout}\scriptsize
\definecolor{shadecolor}{rgb}{0.969, 0.969, 0.969}\color{fgcolor}\begin{kframe}
\begin{alltt}
\hlstd{f0} \hlkwb{<-} \hlkwd{relevel}\hlstd{(f0,} \hlstr{"II"}\hlstd{)}
\hlstd{f0}
\end{alltt}
\begin{verbatim}
##  [1] II  I   III III III I   I   II  I   II 
## Levels: II I III
\end{verbatim}
\end{kframe}
\end{knitrout}

  \end{itemize}
\end{frame}

\begin{frame}[fragile]
  \ft{When Factors Are a PITA}
\begin{knitrout}\scriptsize
\definecolor{shadecolor}{rgb}{0.969, 0.969, 0.969}\color{fgcolor}\begin{kframe}
\begin{alltt}
\hlkwd{set.seed}\hlstd{(}\hlnum{423423}\hlstd{)}
\hlstd{ff} \hlkwb{<-} \hlkwd{factor}\hlstd{(}\hlkwd{sample}\hlstd{(}\hlnum{1}\hlopt{:}\hlnum{4}\hlstd{,} \hlnum{10}\hlstd{,} \hlnum{TRUE}\hlstd{))}
\end{alltt}
\end{kframe}
\end{knitrout}
\begin{knitrout}\scriptsize
\definecolor{shadecolor}{rgb}{0.969, 0.969, 0.969}\color{fgcolor}\begin{kframe}
\begin{alltt}
\hlkwd{mean}\hlstd{(ff)}
\end{alltt}


{\ttfamily\noindent\color{warningcolor}{\#\# Warning in mean.default(ff): argument is not numeric or logical: returning NA}}\begin{verbatim}
## [1] NA
\end{verbatim}
\begin{alltt}
\hlstd{ff} \hlopt{+} \hlnum{10}
\end{alltt}


{\ttfamily\noindent\color{warningcolor}{\#\# Warning in Ops.factor(ff, 10): '+' not meaningful for factors}}\begin{verbatim}
##  [1] NA NA NA NA NA NA NA NA NA NA
\end{verbatim}
\begin{alltt}
\hlkwd{c}\hlstd{(ff,} \hlnum{10}\hlstd{)} \hlcom{# Not a factor anymore}
\end{alltt}
\begin{verbatim}
##  [1]  1  2  1  1  2  1  3  1  2  3 10
\end{verbatim}
\end{kframe}
\end{knitrout}
\end{frame}
  
\begin{frame}[fragile]
  \ft{When Factors Are a PITA}
  
\begin{knitrout}\scriptsize
\definecolor{shadecolor}{rgb}{0.969, 0.969, 0.969}\color{fgcolor}\begin{kframe}
\begin{alltt}
\hlstd{(a} \hlkwb{<-} \hlkwd{factor}\hlstd{(}\hlkwd{sample}\hlstd{(letters,} \hlnum{10}\hlstd{,} \hlkwc{replace} \hlstd{=} \hlnum{TRUE}\hlstd{)))}
\end{alltt}
\begin{verbatim}
##  [1] a u o j i h d g n d
## Levels: a d g h i j n o u
\end{verbatim}
\begin{alltt}
\hlstd{(b} \hlkwb{<-} \hlkwd{factor}\hlstd{(}\hlkwd{sample}\hlstd{(letters,} \hlnum{10}\hlstd{,} \hlkwc{replace} \hlstd{=} \hlnum{TRUE}\hlstd{)))}
\end{alltt}
\begin{verbatim}
##  [1] t y k g v k p b d d
## Levels: b d g k p t v y
\end{verbatim}
\end{kframe}
\end{knitrout}
\begin{knitrout}\scriptsize
\definecolor{shadecolor}{rgb}{0.969, 0.969, 0.969}\color{fgcolor}\begin{kframe}
\begin{alltt}
\hlkwd{c}\hlstd{(a, b)}
\end{alltt}
\begin{verbatim}
##  [1] 1 9 8 6 5 4 2 3 7 2 6 8 4 3 7 4 5 1 2 2
\end{verbatim}
\end{kframe}
\end{knitrout}
\pause
\begin{knitrout}\scriptsize
\definecolor{shadecolor}{rgb}{0.969, 0.969, 0.969}\color{fgcolor}\begin{kframe}
\begin{alltt}
\hlkwd{factor}\hlstd{(}\hlkwd{c}\hlstd{(}\hlkwd{as.character}\hlstd{(a),} \hlkwd{as.character}\hlstd{(b)))}
\end{alltt}
\begin{verbatim}
##  [1] a u o j i h d g n d t y k g v k p b d d
## Levels: a b d g h i j k n o p t u v y
\end{verbatim}
\end{kframe}
\end{knitrout}
\end{frame}

\begin{frame}[fragile]
  \ft{Factors}
  \begin{itemize}
  \item Pros
    \begin{itemize}
    \item Needed for modelling categorical variable
    \item Memory efficient, i.e., factors only need to store values as
      integer and the unique levels as character strings
    \item Nice output
\begin{knitrout}\scriptsize
\definecolor{shadecolor}{rgb}{0.969, 0.969, 0.969}\color{fgcolor}\begin{kframe}
\begin{alltt}
\hlkwd{table}\hlstd{(}\hlkwd{factor}\hlstd{(}\hlkwd{c}\hlstd{(}\hlnum{1}\hlstd{,} \hlnum{2}\hlstd{,} \hlnum{3}\hlstd{),}
             \hlkwc{labels} \hlstd{=} \hlkwd{c}\hlstd{(}\hlstr{"Healthy"}\hlstd{,} \hlstr{"Diseased"}\hlstd{,} \hlstr{"Dead"}\hlstd{)))}
\end{alltt}
\begin{verbatim}
## 
##  Healthy Diseased     Dead 
##        1        1        1
\end{verbatim}
\end{kframe}
\end{knitrout}
    \end{itemize}
  \item Cons
    \begin{itemize}
    \item Require to be cautious for some data manipulation
    \end{itemize}
  \item I'd recommend reading data using the option
    \verb|stringsAsFactors=FALSE| and transform variables into factors
    as needed
  \end{itemize}
\end{frame}

\begin{frame}[fragile]
  \ft{Dates}
  
  R provides several options to deal with dates, which is a
  challenging problem, i.e., time zones, daylight savings, leap
  second, \dots
  
  \begin{itemize}
  \item \verb=as.Date= handles dates without time
  \item The {\bf chron} package handles dates and times, but without
    support for time zones
  \item The {\tt POSIXct} and {\tt POSIXlt} allow for dates and times
    with control for time zones
  \item The {\bf lubridate} packages is supposed to facilitate the use
    of dates and times in R
  \end{itemize}
  {\bf Rule of thumb:} Use the simplest technique possible. If you
  only have dates, use {\tt as.Date}
  
\end{frame}

\begin{frame}[fragile]
  \ft{Dates}
  \fst{{\tt as.Date}}
  
  \begin{itemize}
  \item {\tt as.Date} accepts a variety of input style through the
    {\tt format} argument
  \item Default is {\tt yyyy-mm-dd}
\begin{knitrout}\scriptsize
\definecolor{shadecolor}{rgb}{0.969, 0.969, 0.969}\color{fgcolor}\begin{kframe}
\begin{alltt}
\hlkwd{as.Date}\hlstd{(}\hlstr{"2014-06-12"}\hlstd{)}
\end{alltt}
\begin{verbatim}
## [1] "2014-06-12"
\end{verbatim}
\end{kframe}
\end{knitrout}

\begin{knitrout}\scriptsize
\definecolor{shadecolor}{rgb}{0.969, 0.969, 0.969}\color{fgcolor}\begin{kframe}
\begin{alltt}
\hlkwd{as.Date}\hlstd{(}\hlstr{"12.6.2014"}\hlstd{,} \hlkwc{format} \hlstd{=} \hlstr{"%d.%m.%Y"}\hlstd{)}
\end{alltt}
\begin{verbatim}
## [1] "2014-06-12"
\end{verbatim}
\end{kframe}
\end{knitrout}

\begin{knitrout}\scriptsize
\definecolor{shadecolor}{rgb}{0.969, 0.969, 0.969}\color{fgcolor}\begin{kframe}
\begin{alltt}
\hlkwd{as.Date}\hlstd{(}\hlstr{"12 June 14"}\hlstd{,} \hlkwc{format} \hlstd{=} \hlstr{"%d %B %y"}\hlstd{)}
\end{alltt}
\begin{verbatim}
## [1] "2014-06-12"
\end{verbatim}
\end{kframe}
\end{knitrout}

See {\tt ?strptime} for a complete list of format symbols
  \end{itemize}
  
\end{frame}

\begin{frame}[fragile]
  \ft{Dates}
  \fst{{\tt as.Date}}
  
  \begin{itemize}
  \item Internally, dates are stored as the number of days since
    January 1, 1970
  \item {\tt as.numeric} can be used to convert a date to its numeric form
\begin{knitrout}\small
\definecolor{shadecolor}{rgb}{0.969, 0.969, 0.969}\color{fgcolor}\begin{kframe}
\begin{alltt}
\hlkwd{as.integer}\hlstd{(}\hlkwd{as.Date}\hlstd{(}\hlstr{"2014-06-12"}\hlstd{))}
\end{alltt}
\begin{verbatim}
## [1] 16233
\end{verbatim}
\end{kframe}
\end{knitrout}
  \item The \verb=weekdays= and \verb=months= functions can be used to
    extract the dates' components
  \item Calculation on dates: See \verb=?Ops.Date=. Addition,
    subtraction, logical operations (\verb|==|, \verb=<=, \dots) are
    available

  \end{itemize}
\end{frame}


\section{Text Processing}

\begin{xframe}
  \ft{Introduction}
  
  \begin{description}
  \item[Character:] A symbol in a written language, e.g, letters,
    numbers, punctuation marks, space, newlines, \dots
  \item[String:] A sequence of character bound together
  \end{description}
  Note that R does not distinguish between character and string
\begin{knitrout}\scriptsize
\definecolor{shadecolor}{rgb}{0.969, 0.969, 0.969}\color{fgcolor}\begin{kframe}
\begin{alltt}
\hlstd{test} \hlkwb{<-} \hlstr{"a"} \hlcom{# or 'a'}
\hlstd{test2} \hlkwb{<-} \hlstr{"apple"} \hlcom{# or 'apple'}
\hlkwd{class}\hlstd{(test)}
\end{alltt}
\begin{verbatim}
## [1] "character"
\end{verbatim}
\begin{alltt}
\hlkwd{class}\hlstd{(test2)}
\end{alltt}
\begin{verbatim}
## [1] "character"
\end{verbatim}
\end{kframe}
\end{knitrout}

\end{xframe}


\begin{xframe}
  \ft{Substrings}
  
  The {\tt substr} permits to extract and/or replace substrings
\begin{knitrout}\scriptsize
\definecolor{shadecolor}{rgb}{0.969, 0.969, 0.969}\color{fgcolor}\begin{kframe}
\begin{alltt}
\hlcom{# Extract}
\hlstd{my_string} \hlkwb{<-} \hlstr{"cats don't like dogs"}
\hlkwd{substr}\hlstd{(my_string,} \hlkwc{start} \hlstd{=} \hlnum{6}\hlstd{,} \hlkwc{stop} \hlstd{=} \hlnum{15}\hlstd{)}
\end{alltt}
\begin{verbatim}
## [1] "don't like"
\end{verbatim}
\end{kframe}
\end{knitrout}
\begin{knitrout}\scriptsize
\definecolor{shadecolor}{rgb}{0.969, 0.969, 0.969}\color{fgcolor}\begin{kframe}
\begin{alltt}
\hlcom{# Works with vectors}
\hlstd{my_vector} \hlkwb{<-} \hlkwd{c}\hlstd{(}\hlstr{"cats"}\hlstd{,} \hlstr{"dogs"}\hlstd{,} \hlstr{"apple"}\hlstd{)}
\hlkwd{substr}\hlstd{(my_vector,} \hlnum{2}\hlstd{,} \hlnum{2}\hlstd{)}
\end{alltt}
\begin{verbatim}
## [1] "a" "o" "p"
\end{verbatim}
\end{kframe}
\end{knitrout}
\end{xframe}

\begin{xframe}
  \ft{Split Strings into Vectors}
  
  The {\tt strsplit} function permits to split a string into a list
  containing multiple strings based on a given delimiter
\begin{knitrout}\scriptsize
\definecolor{shadecolor}{rgb}{0.969, 0.969, 0.969}\color{fgcolor}\begin{kframe}
\begin{alltt}
\hlstd{another_string} \hlkwb{<-} \hlstr{"cats, dogs and apples"}
\hlkwd{strsplit}\hlstd{(another_string,} \hlkwc{split} \hlstd{=} \hlstr{","}\hlstd{)}
\end{alltt}
\begin{verbatim}
## [[1]]
## [1] "cats"             " dogs and apples"
\end{verbatim}
\begin{alltt}
\hlkwd{strsplit}\hlstd{(another_string,} \hlkwc{split} \hlstd{=} \hlstr{" "}\hlstd{)[[}\hlnum{1}\hlstd{]]}
\end{alltt}
\begin{verbatim}
## [1] "cats,"  "dogs"   "and"    "apples"
\end{verbatim}
\end{kframe}
\end{knitrout}
\begin{knitrout}\scriptsize
\definecolor{shadecolor}{rgb}{0.969, 0.969, 0.969}\color{fgcolor}\begin{kframe}
\begin{alltt}
\hlstd{yet_another_string} \hlkwb{<-} \hlstr{"walk into a bar"}
\hlkwd{strsplit}\hlstd{(}\hlkwd{c}\hlstd{(another_string, yet_another_string),} \hlkwc{split} \hlstd{=} \hlstr{" "}\hlstd{)}
\end{alltt}
\begin{verbatim}
## [[1]]
## [1] "cats,"  "dogs"   "and"    "apples"
## 
## [[2]]
## [1] "walk" "into" "a"    "bar"
\end{verbatim}
\end{kframe}
\end{knitrout}

\end{xframe}

\begin{xframe}
  \ft{Build Strings from Multiple Parts}
  \fst{The {\tt paste} function}
  
  The {\tt paste} function combines multiple strings into a single
  strings. The {\tt sep} and {\tt collapse} arguments control the
  separation. 
\begin{knitrout}\scriptsize
\definecolor{shadecolor}{rgb}{0.969, 0.969, 0.969}\color{fgcolor}\begin{kframe}
\begin{alltt}
\hlkwd{paste}\hlstd{(}\hlkwd{c}\hlstd{(}\hlstr{"cats"}\hlstd{,} \hlstr{"dogs"}\hlstd{,} \hlstr{"apple"}\hlstd{),} \hlkwc{collapse} \hlstd{=} \hlstr{"|"}\hlstd{)} \hlcom{# BUT}
\end{alltt}
\begin{verbatim}
## [1] "cats|dogs|apple"
\end{verbatim}
\begin{alltt}
\hlkwd{paste}\hlstd{(}\hlkwd{c}\hlstd{(}\hlstr{"cats"}\hlstd{,} \hlstr{"dogs"}\hlstd{,} \hlstr{"apple"}\hlstd{),} \hlkwc{sep} \hlstd{=} \hlstr{"|"}\hlstd{)}
\end{alltt}
\begin{verbatim}
## [1] "cats"  "dogs"  "apple"
\end{verbatim}
\begin{alltt}
\hlcom{# collapse permits to concatenate strings from a single vector}
\end{alltt}
\end{kframe}
\end{knitrout}
\begin{knitrout}\scriptsize
\definecolor{shadecolor}{rgb}{0.969, 0.969, 0.969}\color{fgcolor}\begin{kframe}
\begin{alltt}
\hlkwd{paste}\hlstd{(}\hlstr{"cats"}\hlstd{,} \hlstr{"dogs"}\hlstd{,} \hlstr{"apple"}\hlstd{,} \hlkwc{sep} \hlstd{=} \hlstr{"|"}\hlstd{)}
\end{alltt}
\begin{verbatim}
## [1] "cats|dogs|apple"
\end{verbatim}
\end{kframe}
\end{knitrout}

\end{xframe}

\begin{xframe}
  \ft{Search and Replace}
  
  R provides several functions for searching and replacing text
  \begin{center}
    {\small
      \begin{tabular}{lp{8cm}} 
        \toprule
        {\tt grep}                 & Search for {\tt pattern} in a
        vector {\tt x} and return the indices of matches or matching
        string ({\tt value = TRUE})       \\[5pt]
        {\tt grepl} & As grep but returns a logical vector \\[5pt]
        {\tt regexpr} & Return character position of the first match
        as well as length of the match. -1 is returned if no match \\[5pt]
        {\tt gregexpr} & As {\tt regexpr} but reports all matches \\[5pt]
        {\tt regexec} & Comparable to {\tt regexpr} but returns a list\\[5pt]
        \midrule
        {\tt sub} & Finds pattern in text and replaces first
        match with specified string \\[5pt]
        {\tt gsub} & As {\tt sub} but replaces all matches\\[5pt]
        \bottomrule
      \end{tabular}}
  \end{center}
\end{xframe}

\begin{xframe}
  \ft{Simple Matching}
  
\begin{knitrout}\scriptsize
\definecolor{shadecolor}{rgb}{0.969, 0.969, 0.969}\color{fgcolor}\begin{kframe}
\begin{alltt}
\hlstd{l} \hlkwb{<-} \hlkwd{c}\hlstd{(}\hlstr{"apple"}\hlstd{,} \hlstr{"banana"}\hlstd{,} \hlstr{"grape"}\hlstd{,} \hlstr{"10"}\hlstd{,} \hlstr{"green.pepper"}\hlstd{)}
\hlkwd{grep}\hlstd{(}\hlkwc{pattern} \hlstd{=} \hlstr{"a"}\hlstd{,} \hlkwc{x} \hlstd{= l)}
\end{alltt}
\begin{verbatim}
## [1] 1 2 3
\end{verbatim}
\begin{alltt}
\hlkwd{grep}\hlstd{(}\hlkwc{pattern} \hlstd{=} \hlstr{"a"}\hlstd{,} \hlkwc{x} \hlstd{= l,} \hlkwc{value} \hlstd{=} \hlnum{TRUE}\hlstd{)}
\end{alltt}
\begin{verbatim}
## [1] "apple"  "banana" "grape"
\end{verbatim}
\begin{alltt}
\hlkwd{grepl}\hlstd{(}\hlstr{"a"}\hlstd{, l)}
\end{alltt}
\begin{verbatim}
## [1]  TRUE  TRUE  TRUE FALSE FALSE
\end{verbatim}
\end{kframe}
\end{knitrout}

\end{xframe}


\section{Data Manipulation}


\begin{frame}[fragile]
  \ft{Lists}
  
  Lists are the most general R object. 
\begin{knitrout}\scriptsize
\definecolor{shadecolor}{rgb}{0.969, 0.969, 0.969}\color{fgcolor}\begin{kframe}
\begin{alltt}
\hlstd{(ll} \hlkwb{<-} \hlkwd{list}\hlstd{(}\hlkwc{a} \hlstd{=} \hlnum{1}\hlopt{:}\hlnum{3}\hlstd{,} \hlkwc{b} \hlstd{= month.name[}\hlnum{1}\hlopt{:}\hlnum{5}\hlstd{],} \hlkwc{c} \hlstd{=} \hlkwd{c}\hlstd{(}\hlnum{TRUE}\hlstd{,} \hlnum{FALSE}\hlstd{),}
            \hlkwc{d} \hlstd{=} \hlkwd{data.frame}\hlstd{(}\hlkwc{y} \hlstd{=} \hlkwd{rnorm}\hlstd{(}\hlnum{5}\hlstd{),} \hlkwc{x} \hlstd{=} \hlkwd{rbinom}\hlstd{(}\hlnum{5}\hlstd{,} \hlnum{1}\hlstd{,} \hlnum{.5}\hlstd{))))}
\end{alltt}
\begin{verbatim}
## $a
## [1] 1 2 3
## 
## $b
## [1] "January"  "February" "March"    "April"    "May"     
## 
## $c
## [1]  TRUE FALSE
## 
## $d
##            y x
## 1 -1.1065764 0
## 2  1.6957258 0
## 3 -1.0641906 1
## 4 -0.0415854 1
## 5  0.8534742 0
\end{verbatim}
\end{kframe}
\end{knitrout}
  
\end{frame}

\begin{frame}[fragile]
  \ft{Lists}

\begin{knitrout}\scriptsize
\definecolor{shadecolor}{rgb}{0.969, 0.969, 0.969}\color{fgcolor}\begin{kframe}
\begin{alltt}
\hlkwd{class}\hlstd{(ll[[}\hlnum{4}\hlstd{]]);} \hlkwd{class}\hlstd{(ll[[}\hlstr{"d"}\hlstd{]]);} \hlkwd{class}\hlstd{(ll}\hlopt{$}\hlstd{d)}
\end{alltt}
\begin{verbatim}
## [1] "data.frame"
## [1] "data.frame"
## [1] "data.frame"
\end{verbatim}
\begin{alltt}
\hlkwd{class}\hlstd{(ll[}\hlnum{4}\hlstd{])}
\end{alltt}
\begin{verbatim}
## [1] "list"
\end{verbatim}
\end{kframe}
\end{knitrout}

\begin{knitrout}\scriptsize
\definecolor{shadecolor}{rgb}{0.969, 0.969, 0.969}\color{fgcolor}\begin{kframe}
\begin{alltt}
\hlstd{ll[}\hlkwd{c}\hlstd{(}\hlnum{1}\hlstd{,} \hlnum{3}\hlstd{)]}
\end{alltt}
\begin{verbatim}
## $a
## [1] 1 2 3
## 
## $c
## [1]  TRUE FALSE
\end{verbatim}
\end{kframe}
\end{knitrout}
\end{frame}

\begin{frame}[fragile]
  \ft{Subscripting Data Frames}
  
\begin{knitrout}\scriptsize
\definecolor{shadecolor}{rgb}{0.969, 0.969, 0.969}\color{fgcolor}\begin{kframe}
\begin{alltt}
\hlkwd{set.seed}\hlstd{(}\hlnum{4234234}\hlstd{)}
\hlstd{(df} \hlkwb{<-} \hlkwd{data.frame}\hlstd{(}\hlkwc{x} \hlstd{=} \hlkwd{c}\hlstd{(}\hlkwd{rnorm}\hlstd{(}\hlnum{3}\hlstd{),} \hlnum{NA}\hlstd{,} \hlnum{3}\hlstd{),}
                 \hlkwc{y} \hlstd{=} \hlkwd{c}\hlstd{(}\hlnum{NA}\hlstd{,} \hlkwd{rexp}\hlstd{(}\hlnum{2}\hlstd{,} \hlnum{0.01}\hlstd{),} \hlnum{NA}\hlstd{,} \hlnum{3}\hlstd{)))}
\end{alltt}
\begin{verbatim}
##            x         y
## 1  1.7547348        NA
## 2 -0.3676785 108.34508
## 3 -1.5529115  85.43826
## 4         NA        NA
## 5  3.0000000   3.00000
\end{verbatim}
\begin{alltt}
\hlstd{df}\hlopt{$}\hlstd{x}
\end{alltt}
\begin{verbatim}
## [1]  1.7547348 -0.3676785 -1.5529115         NA  3.0000000
\end{verbatim}
\begin{alltt}
\hlstd{df[,} \hlstr{"x"}\hlstd{,} \hlkwc{drop} \hlstd{=} \hlnum{FALSE}\hlstd{]}
\end{alltt}
\begin{verbatim}
##            x
## 1  1.7547348
## 2 -0.3676785
## 3 -1.5529115
## 4         NA
## 5  3.0000000
\end{verbatim}
\end{kframe}
\end{knitrout}
\end{frame}

\begin{frame}[fragile]
  \ft{Subscripting Data Frames}

\begin{knitrout}\scriptsize
\definecolor{shadecolor}{rgb}{0.969, 0.969, 0.969}\color{fgcolor}\begin{kframe}
\begin{alltt}
\hlstd{df[df}\hlopt{$}\hlstd{y} \hlopt{>} \hlnum{10}\hlstd{, ]}
\end{alltt}
\begin{verbatim}
##               x         y
## NA           NA        NA
## 2    -0.3676785 108.34508
## 3    -1.5529115  85.43826
## NA.1         NA        NA
\end{verbatim}
\end{kframe}
\end{knitrout}

\begin{knitrout}\scriptsize
\definecolor{shadecolor}{rgb}{0.969, 0.969, 0.969}\color{fgcolor}\begin{kframe}
\begin{alltt}
\hlstd{df[}\hlopt{!}\hlkwd{is.na}\hlstd{(df}\hlopt{$}\hlstd{y)} \hlopt{&} \hlstd{df}\hlopt{$}\hlstd{y} \hlopt{>} \hlnum{10}\hlstd{, ]}
\end{alltt}
\begin{verbatim}
##            x         y
## 2 -0.3676785 108.34508
## 3 -1.5529115  85.43826
\end{verbatim}
\end{kframe}
\end{knitrout}

\begin{knitrout}\scriptsize
\definecolor{shadecolor}{rgb}{0.969, 0.969, 0.969}\color{fgcolor}\begin{kframe}
\begin{alltt}
\hlkwd{subset}\hlstd{(df, y} \hlopt{>} \hlnum{10}\hlstd{)}
\end{alltt}
\begin{verbatim}
##            x         y
## 2 -0.3676785 108.34508
## 3 -1.5529115  85.43826
\end{verbatim}
\end{kframe}
\end{knitrout}

\end{frame}

\begin{frame}[fragile]
  \ft{Subscripting Data Frames}

{\small Order a data frame}
\begin{knitrout}\scriptsize
\definecolor{shadecolor}{rgb}{0.969, 0.969, 0.969}\color{fgcolor}\begin{kframe}
\begin{alltt}
\hlstd{df[}\hlkwd{order}\hlstd{(df}\hlopt{$}\hlstd{x), ]}
\end{alltt}
\begin{verbatim}
##            x         y
## 3 -1.5529115  85.43826
## 2 -0.3676785 108.34508
## 1  1.7547348        NA
## 5  3.0000000   3.00000
## 4         NA        NA
\end{verbatim}
\end{kframe}
\end{knitrout}

\end{frame}


\section{Data Aggregation}

\begin{frame}[fragile]
  \ft{Data Aggregation}
  
  \begin{itemize}
  \item For simple tabulation and cross-tabulation, the \verb=table=,
    \verb=ftable= and \verb=xtabs= functions are available
  \item For more complex tasks, the available functions can be
    classified into two groups
    \begin{itemize}
    \item Functions that operate on arrays and/or lists
      (e.g., \verb=*apply=, \verb=sweep=)
    \item Functions oriented towards data frames (e.g.,
      \verb=aggregate=, \verb=by=)
    \end{itemize}
  \end{itemize}
\end{frame}

\begin{frame}[fragile]
  \ft{The {\tt table} function}
  
\begin{knitrout}\scriptsize
\definecolor{shadecolor}{rgb}{0.969, 0.969, 0.969}\color{fgcolor}\begin{kframe}
\begin{alltt}
\hlkwd{data}\hlstd{(iris)}

\hlkwd{head}\hlstd{(iris)}
\end{alltt}
\begin{verbatim}
##   Sepal.Length Sepal.Width Petal.Length Petal.Width Species
## 1          5.1         3.5          1.4         0.2  setosa
## 2          4.9         3.0          1.4         0.2  setosa
## 3          4.7         3.2          1.3         0.2  setosa
## 4          4.6         3.1          1.5         0.2  setosa
## 5          5.0         3.6          1.4         0.2  setosa
## 6          5.4         3.9          1.7         0.4  setosa
\end{verbatim}
\end{kframe}
\end{knitrout}
\end{frame}

\begin{frame}[fragile]
  \ft{The {\tt table} function}
  
  \vspace{-0.5cm}
\begin{knitrout}\scriptsize
\definecolor{shadecolor}{rgb}{0.969, 0.969, 0.969}\color{fgcolor}\begin{kframe}
\begin{alltt}
\hlkwd{table}\hlstd{(iris}\hlopt{$}\hlstd{Species)}
\end{alltt}
\begin{verbatim}
## 
##     setosa versicolor  virginica 
##         50         50         50
\end{verbatim}
\end{kframe}
\end{knitrout}

\begin{knitrout}\scriptsize
\definecolor{shadecolor}{rgb}{0.969, 0.969, 0.969}\color{fgcolor}\begin{kframe}
\begin{alltt}
\hlkwd{table}\hlstd{(iris}\hlopt{$}\hlstd{Species, iris}\hlopt{$}\hlstd{Petal.Length} \hlopt{>} \hlnum{6}\hlstd{)}
\end{alltt}
\begin{verbatim}
##             
##              FALSE TRUE
##   setosa        50    0
##   versicolor    50    0
##   virginica     41    9
\end{verbatim}
\end{kframe}
\end{knitrout}

\begin{knitrout}\scriptsize
\definecolor{shadecolor}{rgb}{0.969, 0.969, 0.969}\color{fgcolor}\begin{kframe}
\begin{alltt}
\hlkwd{as.data.frame}\hlstd{(}\hlkwd{table}\hlstd{(iris}\hlopt{$}\hlstd{Species, iris}\hlopt{$}\hlstd{Petal.Length} \hlopt{>} \hlnum{6}\hlstd{))}
\end{alltt}
\begin{verbatim}
##         Var1  Var2 Freq
## 1     setosa FALSE   50
## 2 versicolor FALSE   50
## 3  virginica FALSE   41
## 4     setosa  TRUE    0
## 5 versicolor  TRUE    0
## 6  virginica  TRUE    9
\end{verbatim}
\end{kframe}
\end{knitrout}
\end{frame}

\begin{frame}[fragile]
  \ft{Road Map for Aggregation}
  
  Three things to consider
  \begin{enumerate}
  \item How are the groups that divide the data defined?
  \item What is the nature of the data to be operated on?
  \item What is the desired end result
  \end{enumerate}
  
\end{frame}

\begin{frame}[fragile]
  \ft{Groups Defined as Lists Elements}
  
  \verb=sapply= or \verb=lapply= are the appropriate functions
  \begin{itemize}
  \item \verb=lapply= always returns a list
  \item \verb=sapply= tries to ``simplify'' the output
  \end{itemize}
\pause
\begin{knitrout}\tiny
\definecolor{shadecolor}{rgb}{0.969, 0.969, 0.969}\color{fgcolor}\begin{kframe}
\begin{alltt}
\hlstd{myList} \hlkwb{<-} \hlkwd{list}\hlstd{()}
\hlkwa{for} \hlstd{(i} \hlkwa{in} \hlnum{1}\hlopt{:}\hlnum{4}\hlstd{) \{}
    \hlstd{myList[[i]]} \hlkwb{<-} \hlkwd{rnorm}\hlstd{(}\hlkwc{n} \hlstd{=} \hlnum{3} \hlopt{*} \hlstd{i)}
\hlstd{\}}
\hlstd{myList}
\end{alltt}
\begin{verbatim}
## [[1]]
## [1]  0.3429579 -0.3193258  0.7808710
## 
## [[2]]
## [1]  1.16866312  0.01419804  0.45813283 -0.43180622  0.34224696 -1.30745260
## 
## [[3]]
## [1]  1.4005004 -1.7575754 -0.2415508  1.0928182 -1.1926425  1.8645074
## [7] -0.3128976  0.8755070 -1.9690762
## 
## [[4]]
##  [1]  1.1415283 -1.1269112 -2.2914106  0.9855559 -1.4959317  2.2773178
##  [7]  0.5160894 -0.5481570 -0.6098171 -1.4302086  1.2207910 -1.7708338
\end{verbatim}
\end{kframe}
\end{knitrout}

\end{frame}

\begin{frame}[fragile]
  \ft{Groups Defined as Lists Elements}

  {\small Both for \verb=lapply= and \verb=sapply=, the first argument
    is a list, the second argument is a function
    
    Third, fourth, \dots arguments are further arguments for the
    function that is applied}
\begin{knitrout}\scriptsize
\definecolor{shadecolor}{rgb}{0.969, 0.969, 0.969}\color{fgcolor}\begin{kframe}
\begin{alltt}
\hlkwd{lapply}\hlstd{(myList, length)}
\end{alltt}
\begin{verbatim}
## [[1]]
## [1] 3
## 
## [[2]]
## [1] 6
## 
## [[3]]
## [1] 9
## 
## [[4]]
## [1] 12
\end{verbatim}
\end{kframe}
\end{knitrout}


\begin{knitrout}\scriptsize
\definecolor{shadecolor}{rgb}{0.969, 0.969, 0.969}\color{fgcolor}\begin{kframe}
\begin{alltt}
\hlkwd{sapply}\hlstd{(myList, length)}
\end{alltt}
\begin{verbatim}
## [1]  3  6  9 12
\end{verbatim}
\end{kframe}
\end{knitrout}

\end{frame}

\begin{frame}[fragile]
  \ft{Groups Defined as Lists Elements}
  
\begin{knitrout}\scriptsize
\definecolor{shadecolor}{rgb}{0.969, 0.969, 0.969}\color{fgcolor}\begin{kframe}
\begin{alltt}
\hlstd{myList[[}\hlnum{2}\hlstd{]][}\hlkwd{c}\hlstd{(}\hlnum{3}\hlstd{,} \hlnum{5}\hlstd{)]} \hlkwb{<-} \hlnum{NA}
\hlkwd{sapply}\hlstd{(myList, mean)}
\end{alltt}
\begin{verbatim}
## [1]  0.26816768          NA -0.02671217 -0.26099896
\end{verbatim}
\end{kframe}
\end{knitrout}

\begin{knitrout}\scriptsize
\definecolor{shadecolor}{rgb}{0.969, 0.969, 0.969}\color{fgcolor}\begin{kframe}
\begin{alltt}
\hlkwd{sapply}\hlstd{(myList, mean,} \hlkwc{na.rm} \hlstd{=} \hlnum{TRUE}\hlstd{)}
\end{alltt}
\begin{verbatim}
## [1]  0.26816768 -0.13909942 -0.02671217 -0.26099896
\end{verbatim}
\end{kframe}
\end{knitrout}

\begin{knitrout}\scriptsize
\definecolor{shadecolor}{rgb}{0.969, 0.969, 0.969}\color{fgcolor}\begin{kframe}
\begin{alltt}
\hlkwd{sapply}\hlstd{(myList, quantile,} \hlkwc{probs} \hlstd{=} \hlkwd{c}\hlstd{(}\hlnum{0.25}\hlstd{,} \hlnum{0.75}\hlstd{),} \hlkwc{na.rm} \hlstd{=} \hlnum{TRUE}\hlstd{)}
\end{alltt}
\begin{verbatim}
##          [,1]       [,2]      [,3]      [,4]
## 25% 0.0118160 -0.6507178 -1.192642 -1.446639
## 75% 0.5619144  0.3028143  1.092818  1.024549
\end{verbatim}
\end{kframe}
\end{knitrout}

\end{frame}

\begin{frame}[fragile]
  \ft{Groups Defined as Lists Elements}
  
\begin{knitrout}\scriptsize
\definecolor{shadecolor}{rgb}{0.969, 0.969, 0.969}\color{fgcolor}\begin{kframe}
\begin{alltt}
\hlstd{mySummary} \hlkwb{<-} \hlkwa{function}\hlstd{(}\hlkwc{x}\hlstd{,} \hlkwc{na.rm} \hlstd{=} \hlnum{FALSE}\hlstd{) \{}
    \hlkwd{data.frame}\hlstd{(}
        \hlkwc{Mean} \hlstd{=} \hlkwd{mean}\hlstd{(x,} \hlkwc{na.rm} \hlstd{= na.rm),}
        \hlkwc{SD} \hlstd{=} \hlkwd{sd}\hlstd{(x,} \hlkwc{na.rm} \hlstd{= na.rm),}
        \hlkwc{Min} \hlstd{=} \hlkwd{min}\hlstd{(x,} \hlkwc{na.rm} \hlstd{= na.rm),}
        \hlkwc{Max} \hlstd{=} \hlkwd{max}\hlstd{(x,} \hlkwc{na.rm} \hlstd{= na.rm))}
\hlstd{\}}

\hlkwd{sapply}\hlstd{(myList, mySummary,} \hlkwc{na.rm} \hlstd{=} \hlnum{TRUE}\hlstd{)}
\end{alltt}
\begin{verbatim}
##      [,1]       [,2]       [,3]        [,4]     
## Mean 0.2681677  -0.1390994 -0.02671217 -0.260999
## SD   0.5538984  1.030286   1.411432    1.446369 
## Min  -0.3193258 -1.307453  -1.969076   -2.291411
## Max  0.780871   1.168663   1.864507    2.277318
\end{verbatim}
\end{kframe}
\end{knitrout}
\end{frame}


\begin{frame}[fragile]
  \ft{Groups Defined by Rows or Columns of a Matrix/Array}
  
  In this case, the {\tt apply} function is the logical choice. 
  
  The \verb=apply= function requires three arguments
  \begin{itemize}
  \item the array/matrix on which to operate
  \item An index telling \verb=apply= which dimension to operate on (\verb=1=
    on rows; \verb=2= on columns, \verb=c(1, 2)= on both
  \item The function to use
  \item Optionally further arguments to be used by the function that
    we want to apply
  \end{itemize}
 
\begin{knitrout}\scriptsize
\definecolor{shadecolor}{rgb}{0.969, 0.969, 0.969}\color{fgcolor}\begin{kframe}
\begin{alltt}
\hlkwd{apply}\hlstd{(iris[,} \hlnum{1}\hlopt{:}\hlnum{4}\hlstd{],} \hlnum{2}\hlstd{, mean)}
\end{alltt}
\begin{verbatim}
## Sepal.Length  Sepal.Width Petal.Length  Petal.Width 
##     5.843333     3.057333     3.758000     1.199333
\end{verbatim}
\end{kframe}
\end{knitrout}
\end{frame}

\begin{frame}[fragile]
  \ft{Groups Defined by Rows or Columns of a Matrix/Array}

\begin{knitrout}\scriptsize
\definecolor{shadecolor}{rgb}{0.969, 0.969, 0.969}\color{fgcolor}\begin{kframe}
\begin{alltt}
\hlkwd{apply}\hlstd{(iris[,} \hlnum{1}\hlopt{:}\hlnum{4}\hlstd{],} \hlnum{2}\hlstd{, mySummary)}
\end{alltt}
\begin{verbatim}
## $Sepal.Length
##       Mean        SD Min Max
## 1 5.843333 0.8280661 4.3 7.9
## 
## $Sepal.Width
##       Mean        SD Min Max
## 1 3.057333 0.4358663   2 4.4
## 
## $Petal.Length
##    Mean       SD Min Max
## 1 3.758 1.765298   1 6.9
## 
## $Petal.Width
##       Mean        SD Min Max
## 1 1.199333 0.7622377 0.1 2.5
\end{verbatim}
\end{kframe}
\end{knitrout}

\end{frame}

\begin{frame}[fragile]
  \ft{Groups Based on One or More Grouping Variables}
  
  A very common operation
  
  A lot of choice in base R + a couple of additional packages that
  facilitates these operations
  
  \begin{itemize}
  \item \verb=aggregate=
  \item \verb=tapply=, \verb=by=
  \item {\em split-apply-combine} strategy
    \begin{itemize}
    \item \verb=split=, \verb=lapply=, \verb=do.call=
    \item {\bf plyr}, {\bf dplyr} package
    \item \dots
    \end{itemize}

  \end{itemize}
\end{frame}

\begin{frame}[fragile]
  \ft{Groups Based on One or More Grouping Variables}
  \fst{{\tt aggregate}}
  
  A natural choice for data summaries of several variables
  \begin{itemize}
  \item First argument: A formula
    \begin{itemize}
    \item LHS: Variables to ``summarise''
    \item RHS: Grouping variables
    \end{itemize}
  \item Second argument: A data frame
  \item Third argument: Function to apply
  \item \dots; Further arguments for {\tt FUN} 
  \end{itemize}

\begin{knitrout}\tiny
\definecolor{shadecolor}{rgb}{0.969, 0.969, 0.969}\color{fgcolor}\begin{kframe}
\begin{alltt}
\hlstd{iris}\hlopt{$}\hlstd{Petal.Length.f} \hlkwb{<-} \hlkwd{factor}\hlstd{(iris}\hlopt{$}\hlstd{Petal.Length} \hlopt{>} \hlnum{4.8}\hlstd{,}
                              \hlkwc{levels} \hlstd{=} \hlkwd{c}\hlstd{(}\hlnum{FALSE}\hlstd{,} \hlnum{TRUE}\hlstd{),}
                              \hlkwc{labels} \hlstd{=} \hlkwd{c}\hlstd{(}\hlstr{"Small petals"}\hlstd{,} \hlstr{"Big petals"}\hlstd{))}
\hlkwd{aggregate}\hlstd{(}\hlkwd{cbind}\hlstd{(Sepal.Length, Sepal.Width)} \hlopt{~} \hlstd{Species} \hlopt{+} \hlstd{Petal.Length.f,}
          \hlkwc{data} \hlstd{= iris,} \hlkwc{FUN} \hlstd{= mean)}
\end{alltt}
\begin{verbatim}
##      Species Petal.Length.f Sepal.Length Sepal.Width
## 1     setosa   Small petals     5.006000    3.428000
## 2 versicolor   Small petals     5.889130    2.765217
## 3  virginica   Small petals     5.700000    2.766667
## 4 versicolor     Big petals     6.475000    2.825000
## 5  virginica     Big petals     6.644681    2.987234
\end{verbatim}
\end{kframe}
\end{knitrout}

\end{frame}

\begin{frame}[fragile]
  \ft{Groups Based on One or More Grouping Variables}
  \fst{Split-Apply-Combine}
  
  Term coined by Hadley Wickham (author of the {\bf ggplot2}, {\bf plyr},
  {\bf reshape}, {\bf dplyr}, \dots, packages)
  \begin{description}
  \item[Split] Divide the problem into smaller pieces
  \item[Apply] Work on each pieces independently
  \item[Combine] Recombine the pieces
  \end{description}
  A common problem for both programming and data analysis; many
  implementations
  \begin{itemize}
  \item In base R: {\tt split()}, {\tt *apply()}, {\tt do.call()}
  \item R-packages: {\bf plyr}, {\bf doBy}, {\bf dplyr}, {\bf
      data.table} (to some extent) 
  \end{itemize}
\end{frame}

\begin{frame}[fragile]
  \ft{Split-Apply-Combine}
  \fst{Base R}
  \begin{itemize}
  \item Split by species
\begin{knitrout}\tiny
\definecolor{shadecolor}{rgb}{0.969, 0.969, 0.969}\color{fgcolor}\begin{kframe}
\begin{alltt}
\hlstd{s_iris} \hlkwb{<-} \hlkwd{split}\hlstd{(iris, iris}\hlopt{$}\hlstd{Species)}

\hlcom{## s_iris is a list with number of items }
\hlcom{## equal to the number of levels of iris$Species}
\hlkwd{length}\hlstd{(s_iris)} \hlopt{==} \hlkwd{length}\hlstd{(}\hlkwd{levels}\hlstd{(iris}\hlopt{$}\hlstd{Species))}
\end{alltt}
\begin{verbatim}
## [1] TRUE
\end{verbatim}
\end{kframe}
\end{knitrout}
\item Apply a function to each item of the list
\begin{knitrout}\tiny
\definecolor{shadecolor}{rgb}{0.969, 0.969, 0.969}\color{fgcolor}\begin{kframe}
\begin{alltt}
\hlstd{s_means} \hlkwb{<-} \hlkwd{lapply}\hlstd{(s_iris,} \hlkwa{function}\hlstd{(}\hlkwc{x}\hlstd{)} \hlkwd{colMeans}\hlstd{(x[}\hlnum{1}\hlopt{:}\hlnum{4}\hlstd{]))}
\hlstd{s_means[[}\hlnum{1}\hlstd{]]}
\end{alltt}
\begin{verbatim}
## Sepal.Length  Sepal.Width Petal.Length  Petal.Width 
##        5.006        3.428        1.462        0.246
\end{verbatim}
\end{kframe}
\end{knitrout}
\item Combine
\begin{knitrout}\tiny
\definecolor{shadecolor}{rgb}{0.969, 0.969, 0.969}\color{fgcolor}\begin{kframe}
\begin{alltt}
\hlstd{(res} \hlkwb{<-} \hlkwd{do.call}\hlstd{(rbind, s_means))}
\end{alltt}
\begin{verbatim}
##            Sepal.Length Sepal.Width Petal.Length Petal.Width
## setosa            5.006       3.428        1.462       0.246
## versicolor        5.936       2.770        4.260       1.326
## virginica         6.588       2.974        5.552       2.026
\end{verbatim}
\end{kframe}
\end{knitrout}

\end{itemize}
  
\end{frame}

%% \begin{frame}[fragile]
%%     \ft{Split-Apply-Combine}
%%     \fst{The {\bf plyr} Package}
    
%%     The *apply functions in base R implement the split-apply-combine
%%     strategy, but are inconsistent
%%     \begin{itemize}
%%     \item \verb=apply()= input {\tt arrays}; split by row and/or columns;
%%       output {\tt array} 
%%     \item \verb=lapply()= input {\tt list} or {\tt vector}; output
%%       {\tt list}
%%     \item \verb=sapply()= input {\tt list} or {\tt vector}; simplify
%%       to vector
%%     \item \verb=tapply= input {\tt data.frame}; output depends
%%     \item \verb=rapply()=, \verb=vapply()=, \verb=mapply()=
%%     \end{itemize}\pause\vspace{0.3cm}
%%     {\bf plyr} brings some consistency: \verb=**ply()=
%%     \begin{description}
%%     \item[first *] Input type ({\tt a} array, {\tt d} data frame, {\tt
%%         l} list)
%%     \item[second *] Output type ({\tt a} array, {\tt d} data frame, {\tt
%%         l} list, \verb=_= discard)
%%     \end{description}
  
%% \end{frame}

%% \begin{frame}[fragile]
%%   \ft{{\bf plyr}}
%%   \fst{{\tt a*ply()}}
  
%%   \verb=y <- a*ply(.data, .margins., .fun, ...)=
%%   \vspace{0.5cm}
%%   \begin{description}
%%   \item[.data] An array
%%   \item[.margins] Subscripts which the function gets applied over
%%   \item[.fun] Function to apply to each piece
%%   \end{description}
%%   Returns an array (\verb|*=a|), a data.frame (\verb|*=d|), a list
%%   (\verb|*=l|)
  
%% \end{frame}

%% \begin{frame}[fragile]
%%   \ft{{\bf plyr}}
%%   \fst{{\tt l*ply()}}
  
%%   \verb=y <- l*ply(.data, .fun, ...)=
%%   \vspace{0.5cm}
%%   \begin{description}
%%   \item[.data] An list
%%   \item[.fun] Function to apply to each item of the list
%%   \end{description}
%%   Returns an array (\verb|*=a|), a data.frame (\verb|*=d|), a list
%%   (\verb|*=l|)
  
%% \end{frame}

%% \begin{frame}[fragile]
%%   \ft{{\bf plyr}}
%%   \fst{{\tt d*ply()}}
  
%%   \verb=y <- d*ply(.data, .variables, .fun, ...)=
%%   \vspace{0.5cm}
%%   \begin{description}
%%   \item[.data] A data frame
%%   \item[.variables] Variables defining the groups
%%   \item[.fun] Function to apply to each group
%%   \end{description}
%%   Returns an array (\verb|*=a|), a data.frame (\verb|*=d|), a list
%%   (\verb|*=l|)
  
%% \end{frame}

%% \begin{frame}[fragile]
%%   \ft{{\bf plyr}}
%%   \fst{{\tt d*ply()}}

%% <<ddply1, size = "scriptsize">>=
%% (res <- ddply(iris, "Species", myLM))
%% @ %def 

%% \pause \vspace{0.5cm} 
%% The only problem with {\bf plyr} is that it
%% is sometimes slow
%% \end{frame}

\begin{frame}
  \ft{The {\bf dplyr} package} 
  
  The {\bf dplyr} package proposes a ``grammar of data manipulation'',
  i.e., it implements ``verbs'' useful for data manipulation.
  
  \begin{description}
  \item[{\tt select}] column subset (select variables)
  \item[{\tt filter}] row subset ($\Leftrightarrow$ {\tt subset} in
    base R)
  \item[mutate] add new/modify rows
  \item[summarise] summary statistics
  \item[{\tt arrange}] re-order the rows
  \item[{\tt do}] arbitrary action
  \end{description}
  
  \begin{itemize}
  \item {\bf dplyr} supports data.frames, data.tables (see later) as
    well as data bases
  \item Operations can be chained using a pipe operator
  \end{itemize}
  
\end{frame}

\begin{frame}[fragile]
  \ft{The {\bf dplyr} package}
  
  Compute the mean sepal width by species for flower whose petal
  length is longer than 4.8
\begin{knitrout}\scriptsize
\definecolor{shadecolor}{rgb}{0.969, 0.969, 0.969}\color{fgcolor}\begin{kframe}
\begin{alltt}
\hlstd{iris} \hlopt \hlkwd{group_by}\hlstd{(Species)} \hlopt \hlkwd{filter}\hlstd{(Petal.Length} \hlopt{>} \hlnum{4.8}\hlstd{)} \hlopt
  \hlkwd{summarise}\hlstd{(}\hlkwc{mean_width} \hlstd{=} \hlkwd{mean}\hlstd{(Petal.Width))}
\end{alltt}
\begin{verbatim}
## Source: local data frame [2 x 2]
## 
##      Species mean_width
##       (fctr)      (dbl)
## 1 versicolor   1.575000
## 2  virginica   2.042553
\end{verbatim}
\end{kframe}
\end{knitrout}
  
\end{frame}

\begin{frame}[fragile]
  \ft{The {\bf data.table} package}
  
  The {\bf data.table} package enhances the base data.frame. The
  package offers (extremely) fast
  
  \begin{itemize}
  \item subset
  \item grouping
  \item update
  \item joints (merging)
  \end{itemize}
  
  A {\tt data.table} inherits from {\tt data.frame}, i.e., it is
  compatible with R functions and packages that only accept
  {\tt data.frame}.
  
\end{frame}

\begin{frame}[fragile]
  \ft{The {\bf data.table} package}
  
  The general syntax is
\begin{knitrout}\scriptsize
\definecolor{shadecolor}{rgb}{0.969, 0.969, 0.969}\color{fgcolor}\begin{kframe}
\begin{alltt}
\hlstd{dt[i, j, by]}
\end{alltt}
\end{kframe}
\end{knitrout}
  \begin{description}
  \item[{\tt i}] permits to select rows (A bit like {\tt subset})
  \item[{\tt j}] permits to update/create columns. Extremely flexible
    {\color{gray}(maybe too much?)}
   \item[{\tt by}] permits to ``group by''
  \end{description}
Additionally, data.tables can be {\em keyed} by one or more variables,
leading to
\begin{itemize}
\item ordered data
\item faster merging by keyed variables
\end{itemize}

\end{frame}

\begin{frame}[fragile]
  \ft{The {\bf data.table} package}
  
  Subset rows in {\tt i}
\begin{knitrout}\scriptsize
\definecolor{shadecolor}{rgb}{0.969, 0.969, 0.969}\color{fgcolor}\begin{kframe}
\begin{alltt}
\hlkwd{require}\hlstd{(data.table)}
\hlstd{(dt_iris} \hlkwb{<-} \hlkwd{data.table}\hlstd{(iris,} \hlkwc{key} \hlstd{=} \hlstr{"Species"}\hlstd{))}
\end{alltt}
\begin{verbatim}
##      Sepal.Length Sepal.Width Petal.Length Petal.Width   Species
##   1:          5.1         3.5          1.4         0.2    setosa
##   2:          4.9         3.0          1.4         0.2    setosa
##   3:          4.7         3.2          1.3         0.2    setosa
##   4:          4.6         3.1          1.5         0.2    setosa
##   5:          5.0         3.6          1.4         0.2    setosa
##  ---                                                            
## 146:          6.7         3.0          5.2         2.3 virginica
## 147:          6.3         2.5          5.0         1.9 virginica
## 148:          6.5         3.0          5.2         2.0 virginica
## 149:          6.2         3.4          5.4         2.3 virginica
## 150:          5.9         3.0          5.1         1.8 virginica
##      Petal.Length.f
##   1:   Small petals
##   2:   Small petals
##   3:   Small petals
##   4:   Small petals
##   5:   Small petals
##  ---               
## 146:     Big petals
## 147:     Big petals
## 148:     Big petals
## 149:     Big petals
## 150:     Big petals
\end{verbatim}
\end{kframe}
\end{knitrout}

\end{frame}

\begin{frame}[fragile]
  \ft{The {\bf data.table} package}
  
  Subset rows in {\tt i}

\begin{knitrout}\scriptsize
\definecolor{shadecolor}{rgb}{0.969, 0.969, 0.969}\color{fgcolor}\begin{kframe}
\begin{alltt}
\hlstd{dt_iris[Species} \hlopt{==} \hlstr{"versicolor"} \hlopt{&} \hlstd{Petal.Length} \hlopt{>} \hlnum{4.8}\hlstd{]}
\end{alltt}
\begin{verbatim}
##    Sepal.Length Sepal.Width Petal.Length Petal.Width    Species
## 1:          6.9         3.1          4.9         1.5 versicolor
## 2:          6.3         2.5          4.9         1.5 versicolor
## 3:          6.7         3.0          5.0         1.7 versicolor
## 4:          6.0         2.7          5.1         1.6 versicolor
##    Petal.Length.f
## 1:     Big petals
## 2:     Big petals
## 3:     Big petals
## 4:     Big petals
\end{verbatim}
\end{kframe}
\end{knitrout}

\end{frame}

\begin{frame}[fragile]
  \ft{The {\bf data.table} package}
  
  Select columns in {\tt j}
\begin{knitrout}\scriptsize
\definecolor{shadecolor}{rgb}{0.969, 0.969, 0.969}\color{fgcolor}\begin{kframe}
\begin{alltt}
\hlstd{dt_iris[}\hlnum{1}\hlopt{:}\hlnum{3}\hlstd{, Species]}
\end{alltt}
\begin{verbatim}
## [1] setosa setosa setosa
## Levels: setosa versicolor virginica
\end{verbatim}
\begin{alltt}
\hlstd{dt_iris[Species} \hlopt{==} \hlstr{"versicolor"} \hlopt{&} \hlstd{Petal.Length} \hlopt{>} \hlnum{4.8}\hlstd{,}
        \hlkwd{list}\hlstd{(Species, Petal.Length)]}
\end{alltt}
\begin{verbatim}
##       Species Petal.Length
## 1: versicolor          4.9
## 2: versicolor          4.9
## 3: versicolor          5.0
## 4: versicolor          5.1
\end{verbatim}
\end{kframe}
\end{knitrout}
\end{frame}

\begin{frame}[fragile]
  \ft{The {\bf data.table} package}
  
  Compute in {\tt j}: As long as {\tt j-expressions} returns a list,
  each element of the list will be converted to a column
  
\begin{knitrout}\scriptsize
\definecolor{shadecolor}{rgb}{0.969, 0.969, 0.969}\color{fgcolor}\begin{kframe}
\begin{alltt}
\hlstd{dt_iris[,} \hlkwd{mean}\hlstd{(Petal.Length)]}
\end{alltt}
\begin{verbatim}
## [1] 3.758
\end{verbatim}
\begin{alltt}
\hlstd{dt_iris[Species} \hlopt{==} \hlstr{"versicolor"}\hlstd{,} \hlkwd{mean}\hlstd{(Petal.Length)]}
\end{alltt}
\begin{verbatim}
## [1] 4.26
\end{verbatim}
\end{kframe}
\end{knitrout}
\end{frame}

\begin{frame}[fragile]
  \ft{The {\bf data.table} package}

\begin{knitrout}\scriptsize
\definecolor{shadecolor}{rgb}{0.969, 0.969, 0.969}\color{fgcolor}\begin{kframe}
\begin{alltt}
\hlstd{dt_iris[Species} \hlopt{==} \hlstr{"versicolor"}\hlstd{,} \hlkwd{list}\hlstd{(}\hlkwc{mean} \hlstd{=} \hlkwd{mean}\hlstd{(Petal.Length),}
        \hlkwc{sd} \hlstd{=} \hlkwd{sd}\hlstd{(Petal.Length))]}
\end{alltt}
\begin{verbatim}
##    mean       sd
## 1: 4.26 0.469911
\end{verbatim}
\begin{alltt}
\hlcom{## With a use defined function}
\hlstd{myFun} \hlkwb{<-} \hlkwa{function}\hlstd{(}\hlkwc{x}\hlstd{) \{}
    \hlkwd{list}\hlstd{(}\hlkwc{mean} \hlstd{=} \hlkwd{mean}\hlstd{(x),}
         \hlkwc{sd} \hlstd{=} \hlkwd{sd}\hlstd{(x))}
\hlstd{\}}
\hlstd{dt_iris[Species} \hlopt{==} \hlstr{"versicolor"}\hlstd{,} \hlkwd{myFun}\hlstd{(Petal.Length)]}
\end{alltt}
\begin{verbatim}
##    mean       sd
## 1: 4.26 0.469911
\end{verbatim}
\end{kframe}
\end{knitrout}

\end{frame}

\begin{frame}[fragile]
  \ft{The {\bf data.table} package}
  
  Group by using {\tt by}
\begin{knitrout}\scriptsize
\definecolor{shadecolor}{rgb}{0.969, 0.969, 0.969}\color{fgcolor}\begin{kframe}
\begin{alltt}
\hlstd{dt_iris[, .N,} \hlkwc{by} \hlstd{=} \hlkwd{list}\hlstd{(Species, Petal.Length.f)]}
\end{alltt}
\begin{verbatim}
##       Species Petal.Length.f  N
## 1:     setosa   Small petals 50
## 2: versicolor   Small petals 46
## 3: versicolor     Big petals  4
## 4:  virginica     Big petals 47
## 5:  virginica   Small petals  3
\end{verbatim}
\begin{alltt}
\hlstd{dt_iris[,} \hlkwd{myFun}\hlstd{(Sepal.Length),} \hlkwc{by} \hlstd{=} \hlkwd{list}\hlstd{(Species, Petal.Length.f)]}
\end{alltt}
\begin{verbatim}
##       Species Petal.Length.f     mean        sd
## 1:     setosa   Small petals 5.006000 0.3524897
## 2: versicolor   Small petals 5.889130 0.5012111
## 3: versicolor     Big petals 6.475000 0.4031129
## 4:  virginica     Big petals 6.644681 0.5955664
## 5:  virginica   Small petals 5.700000 0.7000000
\end{verbatim}
\end{kframe}
\end{knitrout}

\end{frame}

\begin{frame}[fragile]
  \ft{The {\bf data.table} package}

  Reorder the last output by {\tt Species} and {\tt Petal.Length.f}

\begin{knitrout}\scriptsize
\definecolor{shadecolor}{rgb}{0.969, 0.969, 0.969}\color{fgcolor}\begin{kframe}
\begin{alltt}
\hlstd{tmp} \hlkwb{<-} \hlstd{dt_iris[,} \hlkwd{myFun}\hlstd{(Sepal.Length),} \hlkwc{by} \hlstd{=} \hlkwd{list}\hlstd{(Species, Petal.Length.f)]}
\hlstd{(}\hlkwd{setkeyv}\hlstd{(tmp,} \hlkwd{c}\hlstd{(}\hlstr{"Species"}\hlstd{,} \hlstr{"Petal.Length.f"}\hlstd{)))}
\end{alltt}
\begin{verbatim}
##       Species Petal.Length.f     mean        sd
## 1:     setosa   Small petals 5.006000 0.3524897
## 2: versicolor   Small petals 5.889130 0.5012111
## 3: versicolor     Big petals 6.475000 0.4031129
## 4:  virginica   Small petals 5.700000 0.7000000
## 5:  virginica     Big petals 6.644681 0.5955664
\end{verbatim}
\end{kframe}
\end{knitrout}
\end{frame}

\begin{xframe}
  \ft{Update/create column with the {\tt :=} operator}
  
  The {\tt :=} operator adds or update columns by reference
  
\begin{knitrout}\tiny
\definecolor{shadecolor}{rgb}{0.969, 0.969, 0.969}\color{fgcolor}\begin{kframe}
\begin{alltt}
\hlstd{dt_iris[, new_variable} \hlkwb{:=} \hlstd{Petal.Length} \hlopt{+} \hlnum{1}\hlstd{]}
\end{alltt}
\end{kframe}
\end{knitrout}

\end{xframe}

\section{Data Reshaping}

\begin{frame}
  \ft{Data Reshaping}
  An important operation in R
  \begin{itemize}
  \item Most R functions expect their input (usually data frames) to
    be arranged in particular ways
  \item It is the responsibility of the user to ensure that the data
    are in the appropriate form
  \item For instance, data for multiple groups are organised as
    columns, with a column for each group
  \item Most R functions expect values to be in {\bf one} column with
    an additional column specifying the groups  
  \end{itemize}

\end{frame}

\begin{frame}[fragile]
  \ft{Long versus Wide Format} 
  
  Useful concept for, e.g., {\em longitudinal studies}, in which a
  patient may have several measurements over time
  \begin{description}
  \item[Wide] If all the measurements for a single individual are in
    the same row, the data are said to be {\bf wide}
    \small{
\begin{verbatim}
id visit1 visit2
1      90     95
2      80     78
\end{verbatim}}
  \item[Long] If each measurement is in a different row, the data are
    said to be in the {\bf long} format
\small{
\begin{verbatim}
id visit measure
 1     1      90
 1     2      95
 2     1      80
 2     2      78
\end{verbatim}}
  \end{description}
  Most data sets are delivered in the wide format, modelling is done
  in the long format
\end{frame}

\begin{frame}[fragile]
  \ft{The {\tt reshape} Function}
  
  The \verb=reshape= function performs the long $\rightarrow$ wide and
  wide $\rightarrow$ long transformations
  \begin{itemize}
  \item Motivated by longitudinal data (repeated measurements) 
  \item Very flexible function (maybe too much)
  \item {\color{gray} Google very useful for using this function} 
  \end{itemize}
\end{frame}

\begin{frame}[fragile]
  \ft{The {\tt reshape} Function}
  \fst{Wide $\rightarrow$ Long Transformation}
  


As an example, consider a data set on US personal expenditure
\begin{knitrout}\scriptsize
\definecolor{shadecolor}{rgb}{0.969, 0.969, 0.969}\color{fgcolor}\begin{kframe}
\begin{alltt}
\hlstd{usp}
\end{alltt}
\begin{verbatim}
##                  type  X1940  X1945 X1950 X1955 X1960
## 1    Food and Tobacco 22.200 44.500 59.60  73.2 86.80
## 2 Household Operation 10.500 15.500 29.00  36.5 46.20
## 3  Medical and Health  3.530  5.760  9.71  14.0 21.10
## 4       Personal Care  1.040  1.980  2.45   3.4  5.40
## 5   Private Education  0.341  0.974  1.80   2.6  3.64
\end{verbatim}
\end{kframe}
\end{knitrout}
 
\end{frame}

\begin{frame}[fragile]
  \ft{The {\tt reshape} Function}
  \fst{Wide $\rightarrow$ Long Transformation}
  Useful arguments for wide to long transformations
  \begin{itemize}
  \item \verb=varying=: names of sets of variables in the wide format
    that correspond to single variables in long format. Can be a list
    of names (see later)
  \item \verb=v.names=: The name we wish to give the variable
    containing these values in our long dataset
  \item \verb=timevar=: The name we wish to give the variable
    describing the different times or metrics
  \item \verb=times=: the values this variable will have
  \item \verb=idvar=: Values describing the different individuals
  \item \verb=direction=: Character string indicating the direction of
    the transformation; either \verb="wide"= or \verb="long"=
  \item \verb=times=, \verb=split=, \verb=sep=
  \end{itemize}
\end{frame}


\begin{frame}[fragile]
  \ft{The {\tt reshape} Function}
  \fst{Wide $\rightarrow$ Long Transformation}

\begin{knitrout}\scriptsize
\definecolor{shadecolor}{rgb}{0.969, 0.969, 0.969}\color{fgcolor}\begin{kframe}
\begin{alltt}
\hlstd{rr2} \hlkwb{<-} \hlkwd{reshape}\hlstd{(usp,} \hlkwc{varying} \hlstd{=} \hlkwd{list}\hlstd{(}\hlkwd{names}\hlstd{(usp)[}\hlopt{-}\hlnum{1}\hlstd{]),} \hlkwc{idvar} \hlstd{=} \hlstr{"type"}\hlstd{,}
               \hlkwc{times} \hlstd{=} \hlkwd{seq}\hlstd{(}\hlnum{1940}\hlstd{,} \hlnum{1960}\hlstd{,} \hlnum{5}\hlstd{),} \hlkwc{v.names} \hlstd{=} \hlstr{"expenditure"}\hlstd{,}
               \hlkwc{direction} \hlstd{=} \hlstr{"long"}\hlstd{)}
\hlkwd{head}\hlstd{(rr2)}
\end{alltt}
\begin{verbatim}
##                                         type time expenditure
## Food and Tobacco.1940       Food and Tobacco 1940      22.200
## Household Operation.1940 Household Operation 1940      10.500
## Medical and Health.1940   Medical and Health 1940       3.530
## Personal Care.1940             Personal Care 1940       1.040
## Private Education.1940     Private Education 1940       0.341
## Food and Tobacco.1945       Food and Tobacco 1945      44.500
\end{verbatim}
\end{kframe}
\end{knitrout}
\end{frame}

\begin{frame}[fragile]
  \ft{The {\tt reshape} Function}
  \fst{Wide $\rightarrow$ Long Transformation}
\begin{knitrout}\scriptsize
\definecolor{shadecolor}{rgb}{0.969, 0.969, 0.969}\color{fgcolor}\begin{kframe}
\begin{alltt}
\hlstd{rr3} \hlkwb{<-} \hlkwd{reshape}\hlstd{(usp,} \hlkwc{varying} \hlstd{=} \hlkwd{names}\hlstd{(usp)[}\hlopt{-}\hlnum{1}\hlstd{],} \hlkwc{idvar} \hlstd{=} \hlstr{"type"}\hlstd{,}
               \hlkwc{times} \hlstd{=} \hlkwd{seq}\hlstd{(}\hlnum{1940}\hlstd{,} \hlnum{1960}\hlstd{,} \hlnum{5}\hlstd{),} \hlkwc{v.names} \hlstd{=} \hlstr{"expenditure"}\hlstd{,}
               \hlkwc{direction} \hlstd{=} \hlstr{"long"}\hlstd{)}
\hlkwd{head}\hlstd{(rr3)}
\end{alltt}
\begin{verbatim}
##                                         type time expenditure
## Food and Tobacco.1940       Food and Tobacco 1940      22.200
## Household Operation.1940 Household Operation 1940      10.500
## Medical and Health.1940   Medical and Health 1940       3.530
## Personal Care.1940             Personal Care 1940       1.040
## Private Education.1940     Private Education 1940       0.341
## Food and Tobacco.1945       Food and Tobacco 1945      44.500
\end{verbatim}
\end{kframe}
\end{knitrout}

\pause

Specifying a vector of names in \verb=varying= now works because we
also specify how the resulting variable should be named
(\verb=v.names=)
\end{frame}


\begin{frame}[fragile]
  \ft{The {\tt reshape} Function}
  \fst{Wide $\rightarrow$ Long Transformation}
  
  The \verb=sep= argument is sometimes useful to help \verb=reshape=
  automagically find the \verb=v.names=

\begin{knitrout}\scriptsize
\definecolor{shadecolor}{rgb}{0.969, 0.969, 0.969}\color{fgcolor}\begin{kframe}
\begin{alltt}
\hlstd{rr5} \hlkwb{<-} \hlkwd{reshape}\hlstd{(usp,} \hlkwc{varying} \hlstd{=} \hlkwd{names}\hlstd{(usp)[}\hlopt{-}\hlnum{1}\hlstd{],} \hlkwc{idvar} \hlstd{=} \hlstr{"type"}\hlstd{,}
               \hlkwc{sep} \hlstd{=} \hlstr{""}\hlstd{,}
               \hlkwc{direction} \hlstd{=} \hlstr{"long"}\hlstd{)}
\hlkwd{head}\hlstd{(rr5)}
\end{alltt}
\begin{verbatim}
##                                         type time      X
## Food and Tobacco.1940       Food and Tobacco 1940 22.200
## Household Operation.1940 Household Operation 1940 10.500
## Medical and Health.1940   Medical and Health 1940  3.530
## Personal Care.1940             Personal Care 1940  1.040
## Private Education.1940     Private Education 1940  0.341
## Food and Tobacco.1945       Food and Tobacco 1945 44.500
\end{verbatim}
\end{kframe}
\end{knitrout}
\end{frame}

\begin{frame}[fragile]
  \ft{The {\tt reshape} Function}
  \fst{Wide $\rightarrow$ Long Transformation}
  
  Reshape()'d data have additional attributes so that the inverse
  transformation is easy 
\begin{knitrout}\scriptsize
\definecolor{shadecolor}{rgb}{0.969, 0.969, 0.969}\color{fgcolor}\begin{kframe}
\begin{alltt}
\hlkwd{reshape}\hlstd{(rr2)}
\end{alltt}
\begin{verbatim}
##                                         type  X1940  X1945 X1950 X1955
## Food and Tobacco.1940       Food and Tobacco 22.200 44.500 59.60  73.2
## Household Operation.1940 Household Operation 10.500 15.500 29.00  36.5
## Medical and Health.1940   Medical and Health  3.530  5.760  9.71  14.0
## Personal Care.1940             Personal Care  1.040  1.980  2.45   3.4
## Private Education.1940     Private Education  0.341  0.974  1.80   2.6
##                          X1960
## Food and Tobacco.1940    86.80
## Household Operation.1940 46.20
## Medical and Health.1940  21.10
## Personal Care.1940        5.40
## Private Education.1940    3.64
\end{verbatim}
\end{kframe}
\end{knitrout}

\end{frame}

\begin{frame}[fragile]
  \ft{The {\tt reshape} Function}
  \fst{Long $\rightarrow$ Wide Transformation}

\begin{knitrout}\scriptsize
\definecolor{shadecolor}{rgb}{0.969, 0.969, 0.969}\color{fgcolor}\begin{kframe}
\begin{alltt}
\hlstd{longdat} \hlkwb{<-} \hlkwd{data.frame}\hlstd{(}\hlkwc{id} \hlstd{=} \hlkwd{as.integer}\hlstd{(}\hlkwd{mapply}\hlstd{(rep,} \hlnum{1}\hlopt{:}\hlnum{3}\hlstd{,} \hlnum{3}\hlstd{)),}
                      \hlkwc{visit} \hlstd{=} \hlkwd{rep}\hlstd{(}\hlnum{1}\hlopt{:}\hlnum{3}\hlstd{,} \hlnum{3}\hlstd{),}
                      \hlkwc{x} \hlstd{=} \hlkwd{rnorm}\hlstd{(}\hlnum{9}\hlstd{),} \hlkwc{y} \hlstd{=} \hlkwd{rnorm}\hlstd{(}\hlnum{9}\hlstd{))}
\hlstd{longdat}
\end{alltt}
\begin{verbatim}
##   id visit           x          y
## 1  1     1 -0.38732169 -1.9495348
## 2  1     2  0.82067539  1.9460650
## 3  1     3 -0.49831634  0.8050138
## 4  2     1 -0.80859026  1.0224776
## 5  2     2 -1.05940918  1.3847261
## 6  2     3 -0.01233044  1.3253185
## 7  3     1  0.84289345 -1.1346494
## 8  3     2  1.56222152  0.3455885
## 9  3     3 -1.44585913 -1.8670554
\end{verbatim}
\end{kframe}
\end{knitrout}

\end{frame}

\begin{frame}[fragile]
  \ft{The {\tt reshape} Function}
  \fst{Long $\rightarrow$ Wide Transformation}
  
  Arguments needed (beside the data set to reshape)
  \begin{itemize}
  \item \verb=idvar=: names of variable that define the experimental
    units
  \item \verb=v.names=: Variables that are used to create the multiple
    variables in the wide format
  \item \verb=timevar= identifies the ``time'' variable for the
    repeated measurements
  \item \verb=direction=: \verb="long"= or \verb="wide"=
  \end{itemize}

\end{frame}

\begin{frame}[fragile]
  \ft{The {\tt reshape} Function}
  \fst{Long $\rightarrow$ Wide Transformation}
  
\begin{knitrout}\scriptsize
\definecolor{shadecolor}{rgb}{0.969, 0.969, 0.969}\color{fgcolor}\begin{kframe}
\begin{alltt}
\hlstd{widedat} \hlkwb{<-} \hlkwd{reshape}\hlstd{(longdat,} \hlkwc{idvar} \hlstd{=} \hlstr{"id"}\hlstd{,} \hlkwc{v.names} \hlstd{=} \hlkwd{c}\hlstd{(}\hlstr{"x"}\hlstd{,} \hlstr{"y"}\hlstd{),}
                   \hlkwc{timevar} \hlstd{=} \hlstr{"visit"}\hlstd{,} \hlkwc{direction} \hlstd{=} \hlstr{"wide"}\hlstd{)}
\hlstd{widedat}
\end{alltt}
\begin{verbatim}
##   id        x.1       y.1        x.2       y.2         x.3        y.3
## 1  1 -0.3873217 -1.949535  0.8206754 1.9460650 -0.49831634  0.8050138
## 4  2 -0.8085903  1.022478 -1.0594092 1.3847261 -0.01233044  1.3253185
## 7  3  0.8428935 -1.134649  1.5622215 0.3455885 -1.44585913 -1.8670554
\end{verbatim}
\end{kframe}
\end{knitrout}
\end{frame}

\begin{frame}[fragile]
  \ft{The {\tt reshape} Function}
  \fst{Long $\rightarrow$ Wide Transformation}

  Wide to long transformation again easy from the reshape()'d data
\begin{knitrout}\scriptsize
\definecolor{shadecolor}{rgb}{0.969, 0.969, 0.969}\color{fgcolor}\begin{kframe}
\begin{alltt}
\hlkwd{reshape}\hlstd{(widedat)}
\end{alltt}
\begin{verbatim}
##     id visit           x          y
## 1.1  1     1 -0.38732169 -1.9495348
## 2.1  2     1 -0.80859026  1.0224776
## 3.1  3     1  0.84289345 -1.1346494
## 1.2  1     2  0.82067539  1.9460650
## 2.2  2     2 -1.05940918  1.3847261
## 3.2  3     2  1.56222152  0.3455885
## 1.3  1     3 -0.49831634  0.8050138
## 2.3  2     3 -0.01233044  1.3253185
## 3.3  3     3 -1.44585913 -1.8670554
\end{verbatim}
\end{kframe}
\end{knitrout}

\end{frame}



\section{Combining and Merging}

\begin{frame}[fragile]
  \ft{Combining Data Frames}
  
  At the most basic level, two or more data frames can be combined by
  rows using \verb=rbind=, or by columns using \verb=cbind=
  \begin{description}
  \item[rbind] Data frames must have the same number of columns
  \item[cbind] The data must have the same number of rows
  \end{description} 
\begin{knitrout}\scriptsize
\definecolor{shadecolor}{rgb}{0.969, 0.969, 0.969}\color{fgcolor}\begin{kframe}
\begin{alltt}
\hlstd{d1} \hlkwb{<-} \hlkwd{data.frame}\hlstd{(}\hlkwc{x} \hlstd{=} \hlkwd{rnorm}\hlstd{(}\hlnum{4}\hlstd{),} \hlkwc{y} \hlstd{=} \hlkwd{sample}\hlstd{(letters,} \hlnum{4}\hlstd{,} \hlkwc{replace} \hlstd{=} \hlnum{FALSE}\hlstd{))}
\hlstd{d2} \hlkwb{<-} \hlkwd{data.frame}\hlstd{(}\hlkwc{x} \hlstd{=} \hlkwd{rnorm}\hlstd{(}\hlnum{4}\hlstd{),} \hlkwc{y} \hlstd{=} \hlkwd{sample}\hlstd{(letters,} \hlnum{4}\hlstd{,} \hlkwc{replace} \hlstd{=} \hlnum{FALSE}\hlstd{))}
\hlstd{d1}
\end{alltt}
\begin{verbatim}
##            x y
## 1 -2.1928785 s
## 2  0.7911268 u
## 3 -1.1763532 l
## 4  1.2151311 b
\end{verbatim}
\end{kframe}
\end{knitrout}

\end{frame}

\begin{frame}[fragile]
  \ft{Combining Data Frames}  
  \fst{{\tt cbind}}
\begin{knitrout}\scriptsize
\definecolor{shadecolor}{rgb}{0.969, 0.969, 0.969}\color{fgcolor}\begin{kframe}
\begin{alltt}
\hlkwd{cbind}\hlstd{(d1, d2)}
\end{alltt}
\begin{verbatim}
##            x y           x y
## 1 -2.1928785 s -0.09692773 m
## 2  0.7911268 u -1.88955642 l
## 3 -1.1763532 l  0.50479514 b
## 4  1.2151311 b  1.15345513 e
\end{verbatim}
\end{kframe}
\end{knitrout}
{\small Duplicate column names are not detected}
\pause 
\begin{knitrout}\scriptsize
\definecolor{shadecolor}{rgb}{0.969, 0.969, 0.969}\color{fgcolor}\begin{kframe}
\begin{alltt}
\hlkwd{cbind}\hlstd{(d1,} \hlkwc{z} \hlstd{=} \hlkwd{c}\hlstd{(}\hlnum{1}\hlstd{,} \hlnum{2}\hlstd{))}
\end{alltt}
\begin{verbatim}
##            x y z
## 1 -2.1928785 s 1
## 2  0.7911268 u 2
## 3 -1.1763532 l 1
## 4  1.2151311 b 2
\end{verbatim}
\end{kframe}
\end{knitrout}
{\small Smaller vectors/data are recycled}
\end{frame}

\begin{frame}[fragile]
  \ft{Combining Data Frames}  
  \fst{{\tt rbind}}
  
  For using \verb=rbind=, names and classes of values to be joined
  must match 
\begin{knitrout}\scriptsize
\definecolor{shadecolor}{rgb}{0.969, 0.969, 0.969}\color{fgcolor}\begin{kframe}
\begin{alltt}
\hlkwd{rbind}\hlstd{(d1, d2)}
\end{alltt}
\begin{verbatim}
##             x y
## 1 -2.19287849 s
## 2  0.79112680 u
## 3 -1.17635317 l
## 4  1.21513114 b
## 5 -0.09692773 m
## 6 -1.88955642 l
## 7  0.50479514 b
## 8  1.15345513 e
\end{verbatim}
\end{kframe}
\end{knitrout}
  
\end{frame}

\begin{frame}[fragile]
  \ft{Combining Data Frames}  
  \fst{{\tt rbind}}
  
\begin{knitrout}\scriptsize
\definecolor{shadecolor}{rgb}{0.969, 0.969, 0.969}\color{fgcolor}\begin{kframe}
\begin{alltt}
\hlstd{d1}\hlopt{$}\hlstd{y} \hlkwb{<-} \hlkwd{factor}\hlstd{(d1}\hlopt{$}\hlstd{y)}
\hlstd{d2}\hlopt{$}\hlstd{y} \hlkwb{<-} \hlkwd{factor}\hlstd{(d2}\hlopt{$}\hlstd{y)}
\hlkwd{rbind}\hlstd{(d1, d2)}
\end{alltt}
\begin{verbatim}
##             x y
## 1 -2.19287849 s
## 2  0.79112680 u
## 3 -1.17635317 l
## 4  1.21513114 b
## 5 -0.09692773 m
## 6 -1.88955642 l
## 7  0.50479514 b
## 8  1.15345513 e
\end{verbatim}
\end{kframe}
\end{knitrout}
It works!
\end{frame}

\begin{frame}[fragile]
  \ft{Combining Data Frames}  
  \fst{{\tt rbind}}
\begin{knitrout}\tiny
\definecolor{shadecolor}{rgb}{0.969, 0.969, 0.969}\color{fgcolor}\begin{kframe}
\begin{alltt}
\hlstd{(d3} \hlkwb{<-} \hlkwd{rbind}\hlstd{(d1,} \hlkwd{data.frame}\hlstd{(}\hlkwc{x} \hlstd{=} \hlstr{"X"}\hlstd{,} \hlkwc{y} \hlstd{=} \hlnum{12}\hlstd{)))}
\end{alltt}


{\ttfamily\noindent\color{warningcolor}{\#\# Warning in `[<-.factor`(`*tmp*`, ri, value = structure(c(3L, 4L, 2L, 1L, : invalid factor level, NA generated}}\begin{verbatim}
##                   x    y
## 1 -2.19287849290542    s
## 2 0.791126796699532    u
## 3 -1.17635316755859    l
## 4   1.2151311364279    b
## 5                 X <NA>
\end{verbatim}
\begin{alltt}
\hlkwd{sapply}\hlstd{(d3, class)}
\end{alltt}
\begin{verbatim}
##           x           y 
## "character"    "factor"
\end{verbatim}
\end{kframe}
\end{knitrout}
\begin{knitrout}\scriptsize
\definecolor{shadecolor}{rgb}{0.969, 0.969, 0.969}\color{fgcolor}\begin{kframe}
\begin{alltt}
\hlkwd{rbind}\hlstd{(d1,} \hlkwd{data.frame}\hlstd{(}\hlkwc{y} \hlstd{=} \hlstr{"X"}\hlstd{,} \hlkwc{d} \hlstd{=} \hlnum{12}\hlstd{))}
\end{alltt}


{\ttfamily\noindent\bfseries\color{errorcolor}{\#\# Error in match.names(clabs, names(xi)): names do not match previous names}}\end{kframe}
\end{knitrout}

\end{frame}

\begin{frame}[fragile]
  \ft{Merge Data Frames}
  
  For more complicated tasks, the \verb=merge= function can be used
  \begin{itemize}
  \item The default behaviour of merge is to join together rows of the
    data frames based on the values of all of the variables (columns)
    that the data frames have in common ({\em natural join})
  \item When called without argument, \verb=merge= only returns rows
    which have observations in both data frames
\begin{knitrout}\scriptsize
\definecolor{shadecolor}{rgb}{0.969, 0.969, 0.969}\color{fgcolor}\begin{kframe}
\begin{alltt}
\hlstd{dd1} \hlkwb{<-} \hlkwd{data.frame}\hlstd{(}\hlkwc{a} \hlstd{=} \hlkwd{c}\hlstd{(}\hlnum{1}\hlstd{,}\hlnum{2}\hlstd{,}\hlnum{4}\hlstd{,}\hlnum{5}\hlstd{,}\hlnum{6}\hlstd{),} \hlkwc{x} \hlstd{=} \hlkwd{c}\hlstd{(}\hlnum{9}\hlstd{,}\hlnum{12}\hlstd{,}\hlnum{14}\hlstd{,}\hlnum{21}\hlstd{,}\hlnum{8}\hlstd{))}
\hlstd{dd2} \hlkwb{<-} \hlkwd{data.frame}\hlstd{(}\hlkwc{a}\hlstd{=}\hlkwd{c}\hlstd{(}\hlnum{1}\hlstd{,}\hlnum{3}\hlstd{,}\hlnum{4}\hlstd{,}\hlnum{6}\hlstd{),}\hlkwc{y}\hlstd{=}\hlkwd{c}\hlstd{(}\hlnum{8}\hlstd{,}\hlnum{14}\hlstd{,}\hlnum{19}\hlstd{,}\hlnum{2}\hlstd{))}
\hlkwd{merge}\hlstd{(dd1, dd2)}
\end{alltt}
\begin{verbatim}
##   a  x  y
## 1 1  9  8
## 2 4 14 19
## 3 6  8  2
\end{verbatim}
\end{kframe}
\end{knitrout}
  \end{itemize}
\end{frame}

\begin{frame}[fragile]
  \ft{Merge Data Frames}
  
  To change the default behaviour the arguments
  \begin{itemize}
  \item \verb|all = TRUE|: Includes all rows ({\em full outer join})
  \item \verb|all.x = TRUE|: Includes all rows of the first data frame
    ({\em left outer join})
  \item \verb|all.y = TRUE|: Includes all rows of the second data frame
    ({\em right outer join})
  \end{itemize}

\end{frame}

\begin{frame}[fragile]
  \ft{Merge Data Frames}
  \begin{itemize}
  \item The \verb=by= argument permits to specify the name of the
    variables that should be used for the merging.
  \item If the merging variables have different names in the data
    frames to merge, the \verb=by.x= and \verb=by.y= arguments can be
    used
  \end{itemize} 
\begin{knitrout}\scriptsize
\definecolor{shadecolor}{rgb}{0.969, 0.969, 0.969}\color{fgcolor}\begin{kframe}
\begin{alltt}
\hlstd{dd1}\hlopt{$}\hlstd{PAT} \hlkwb{<-} \hlstd{letters[}\hlnum{1}\hlopt{:}\hlnum{5}\hlstd{]}
\hlstd{dd2}\hlopt{$}\hlstd{id} \hlkwb{<-} \hlstd{letters[}\hlnum{3}\hlopt{:}\hlnum{6}\hlstd{]}
\hlkwd{merge}\hlstd{(dd1, dd2,} \hlkwc{by.x} \hlstd{=} \hlkwd{c}\hlstd{(}\hlstr{"PAT"}\hlstd{),} \hlkwc{by.y} \hlstd{=} \hlkwd{c}\hlstd{(}\hlstr{"id"}\hlstd{))}
\end{alltt}
\begin{verbatim}
##   PAT a.x  x a.y  y
## 1   c   4 14   1  8
## 2   d   5 21   3 14
## 3   e   6  8   4 19
\end{verbatim}
\end{kframe}
\end{knitrout}
\pause
Note the new variables \verb=a.x= and \verb=a.y=
  
\end{frame}

\begin{xframe}
  \ft{Merge with {\bf dplyr}}
  
  {\bf dplyr} includes some functions for merging data sets
  \begin{description}
  \item[{\tt inner\_join}] Equivalent to {\tt merge} without arguments
  \item[{\tt left\_join}] Equivalent to {\tt merge} with {\tt all.x = TRUE}
  \item[{\tt right\_join}] Equivalent to {\tt merge} with {\tt all.y = TRUE}
  \item[{\tt full\_join}] Equivalent to {\tt merge} with {\tt all = TRUE}
  \end{description}

\begin{knitrout}\scriptsize
\definecolor{shadecolor}{rgb}{0.969, 0.969, 0.969}\color{fgcolor}\begin{kframe}
\begin{alltt}
\hlkwd{left_join}\hlstd{(dd1, dd2,} \hlkwc{by} \hlstd{=} \hlkwd{c}\hlstd{(}\hlstr{"PAT"} \hlstd{=} \hlstr{"id"}\hlstd{))}
\end{alltt}
\begin{verbatim}
##   a.x  x PAT a.y  y
## 1   1  9   a  NA NA
## 2   2 12   b  NA NA
## 3   4 14   c   1  8
## 4   5 21   d   3 14
## 5   6  8   e   4 19
\end{verbatim}
\end{kframe}
\end{knitrout}

\end{xframe}

\begin{xframe}
  \ft{Merge with {\bf data.table}}
  
  \begin{itemize}
  \item A {\tt merge} function is available in the {\bf data.table}
    package.
  \item It works in the same way as the base function
  \end{itemize}
\end{xframe}


\end{document}
