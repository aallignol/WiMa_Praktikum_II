\documentclass[a4,11pt]{article}\usepackage[]{graphicx}\usepackage[]{color}
%% maxwidth is the original width if it is less than linewidth
%% otherwise use linewidth (to make sure the graphics do not exceed the margin)
\makeatletter
\def\maxwidth{ %
  \ifdim\Gin@nat@width>\linewidth
    \linewidth
  \else
    \Gin@nat@width
  \fi
}
\makeatother

\definecolor{fgcolor}{rgb}{0.345, 0.345, 0.345}
\newcommand{\hlnum}[1]{\textcolor[rgb]{0.686,0.059,0.569}{#1}}%
\newcommand{\hlstr}[1]{\textcolor[rgb]{0.192,0.494,0.8}{#1}}%
\newcommand{\hlcom}[1]{\textcolor[rgb]{0.678,0.584,0.686}{\textit{#1}}}%
\newcommand{\hlopt}[1]{\textcolor[rgb]{0,0,0}{#1}}%
\newcommand{\hlstd}[1]{\textcolor[rgb]{0.345,0.345,0.345}{#1}}%
\newcommand{\hlkwa}[1]{\textcolor[rgb]{0.161,0.373,0.58}{\textbf{#1}}}%
\newcommand{\hlkwb}[1]{\textcolor[rgb]{0.69,0.353,0.396}{#1}}%
\newcommand{\hlkwc}[1]{\textcolor[rgb]{0.333,0.667,0.333}{#1}}%
\newcommand{\hlkwd}[1]{\textcolor[rgb]{0.737,0.353,0.396}{\textbf{#1}}}%

\usepackage{framed}
\makeatletter
\newenvironment{kframe}{%
 \def\at@end@of@kframe{}%
 \ifinner\ifhmode%
  \def\at@end@of@kframe{\end{minipage}}%
  \begin{minipage}{\columnwidth}%
 \fi\fi%
 \def\FrameCommand##1{\hskip\@totalleftmargin \hskip-\fboxsep
 \colorbox{shadecolor}{##1}\hskip-\fboxsep
     % There is no \\@totalrightmargin, so:
     \hskip-\linewidth \hskip-\@totalleftmargin \hskip\columnwidth}%
 \MakeFramed {\advance\hsize-\width
   \@totalleftmargin\z@ \linewidth\hsize
   \@setminipage}}%
 {\par\unskip\endMakeFramed%
 \at@end@of@kframe}
\makeatother

\definecolor{shadecolor}{rgb}{.97, .97, .97}
\definecolor{messagecolor}{rgb}{0, 0, 0}
\definecolor{warningcolor}{rgb}{1, 0, 1}
\definecolor{errorcolor}{rgb}{1, 0, 0}
\newenvironment{knitrout}{}{} % an empty environment to be redefined in TeX

\usepackage{alltt}

%% \title{Analysis of the {\ttfamily GSOEP9402} data}
%% \author{Arthur Allignol}
%% \date{}

\usepackage{booktabs}
\usepackage[osf]{sourcesanspro}
\usepackage{sourcecodepro}
\usepackage[T1]{fontenc}
%\renewcommand*{\ttdefault}{sourcecodepro}
\usepackage{subcaption}
\usepackage{graphicx,rotating,epsfig,multirow,multicol,hhline}
\usepackage{amsmath,amsthm,amssymb,amsfonts}
\usepackage{url}
\usepackage{hyperref}

\renewcommand{\familydefault}{\sfdefault}
\usepackage{sfmath}


\def\bea{\begin{eqnarray*}}
\def\eea{\end{eqnarray*}}
%Dense format
\setlength{\parindent}{0em} \setlength{\textwidth}{16cm} \setlength{\textheight}{25cm}
\setlength{\topmargin}{-0.5cm} \setlength{\oddsidemargin}{0cm} \setlength{\headheight}{0cm}
\setlength{\headsep}{0cm}
\newcommand{\nat}{{\it I\hspace{-0.12cm}N}}
\pagestyle{empty}


\newcommand{\p}{{\rm P}}
\newcommand{\erw}{{\rm E}}
\newcommand{\var}{{\rm var}}
\newcommand{\eins}{{\bf 1}}%{\mathbf{1}}
\newcommand{\dif}{{\rm d}}
\newcommand{\cif}{{\rm CIF}}
\newcommand{\Prob}{\mathbb{P}}
\newcommand{\R}{\mathbb{R}}
\newcommand{\D}{\mathrm{d}}

%Special commands:
\newcommand{\Bin}{\operatorname{Bin}} % Binomial Distribution
\newcommand{\NegBin}{\operatorname{NegBin}} % Negative Bin
\newcommand{\HypGeom}{\operatorname{HypGeom}} % Hypergeometric Distribution
\newcommand{\Pois}{\operatorname{Po}} % Hypergeometric Distribution
\newcommand{\Po}{\operatorname{Po}} %
\newcommand{\Exp}{\operatorname{Exp}} %
\newcommand{\Par}{\operatorname{Par}} %
\newcommand{\Ga}{\mathcal{G}a} %
\newcommand{\Be}{\mathcal{B}e} %
\newcommand{\Var}{\operatorname{Var}} %
\newcommand{\E}{\operatorname{E}} %
\newcommand{\Cov}{\operatorname{Cov}} %
\newcommand{\MSE}{\operatorname{MSE}}

\DeclareMathOperator{\Nor}{N} % Normal -
\DeclareMathOperator{\Log}{Log} % Logistische Verteilung -
\newcommand{\ml}[2][1]{% % für Maximum-Likelihood-Schätzer von #1
\ifthenelse{#1 = 1}%
 {\hat{#2}_{\scriptscriptstyle{ML}}}%
 {\hat{#2}^{#1}_{\scriptscriptstyle{ML}}}% z.B. für sigmadach^2
}

%%%%%%%%%%%%%%%%%%%%%%%%%%%%%%%%%%%%%%%%%%%%%%%%%%%%%%%%%%%%%%%%%%%%%%
% Environment for Aufgaben
%%%%%%%%%%%%%%%%%%%%%%%%%%%%%%%%%%%%%%%%%%%%%%%%%%%%%%%%%%%%%%%%%%%%%%

\newcounter{ka}


\newtheorem{exercise}{Problem}
\newenvironment{aufgabe}{\begin{exercise}\sf}{\end{exercise} \bigskip}

\newcommand{\footer}{ \vfill
  \mbox{}\hrulefill\\
  Upload your solutions until the 9.6.2016, 10h00 s.t. on
  Moodle. Only one solution per group. Please upload the source files
  as well as the solutions.}
%\xxivtime   Uhrzeit für letzte Änderung

\pagestyle{empty}

\renewcommand{\labelenumi}{(\alph{enumi})}
\IfFileExists{upquote.sty}{\usepackage{upquote}}{}
\begin{document}

\renewcommand{\baselinestretch}{1}

\hrulefill\\
{\bf WiMa-Praktikum II -- Stochastic} \hspace{\fill} Summer term 2016\\
Arthur Allignol \hspace{\fill} 2.6.2016\\[-1.2ex]
\mbox{}\hrulefill\\
\newline \renewcommand{\baselinestretch}{1}
\setcounter{ka}{0} \vspace{-0.5cm}



\begin{center}
\large{{\bf Exercise Sheet 4 --- Linear Model}}
\end{center}



\begin{aufgabe}{{\bf  Discrimination in Salaries}}
  The data set {\tt salaries.csv} contains data on the salaries of 52
  tenure-track faculty in a small college (Weisberg, 1985). The
  question is to assess whether women professors are paid less then
  their male counterparts in comparable positions. The variables are
  \begin{description}
  \item[{\tt sx}] Gender
  \item[{\tt rk}] Rank coded as 1 for assistant professor, 2 for associate
    professor and 3 for full professor
  \item[{\tt yr}] Number of years at current rank
  \item[{\tt dg}] Highest degree. 1 for doctorate, 0 for masters
  \item[{\tt yd}] Number of years since highest degree was earned
  \item[{\tt sl}] Academic year salary, in dollars 
  \end{description}

  \begin{enumerate}
  \item Fit a linear model with {\tt sl} as response and gender as
    explanatory variables.  Is there evidence of gender
    discrimination? 
  \item Display a ``residuals versus fitted values'' plot. Interpret.
  \item For answering the question we need to control for differences
    in qualifications and experience. We could adjust for either rank,
    years at rank and its interaction, or for highest degree, years
    since the degree was earned and its interaction. Fit these two
    models and comment. Do we need the interaction terms? If there is
    also discrimination in promotion, which model is likely to be
    misleading?
  \item Perform regression diagnostics on the two models of question
    c).
  \item A transformation of the response $g(y)$ is sometimes a good
    remedy for fixing problems with the residuals. Regress the log of
    salary on the variable gender. Describe the gender effect on
    {\em salary} (not log-salary.)  

  \end{enumerate}

\end{aufgabe}

\begin{aufgabe}

  The file {\tt df.csv} contains a simulated data set with variables
  \begin{description}
  \item[{\tt y}] response variable (continuous)
  \item[{\tt x}] an explanatory variable (continuous)
  \item[{\tt treatment}] an explanatory variable (ordinal with levels
    0, 1, 2)
  \end{description}
  Fit a linear model including {\tt x} and {\tt treatment} as
  explanatory variables. Plot the residuals vs fitted values as colour
  the points according to {\tt treatment}. What does the graphic
  suggests? 
\end{aufgabe}

\footer


\end{document}
