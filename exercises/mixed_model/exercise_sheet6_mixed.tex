\documentclass[a4,11pt]{article}\usepackage[]{graphicx}\usepackage[]{color}
%% maxwidth is the original width if it is less than linewidth
%% otherwise use linewidth (to make sure the graphics do not exceed the margin)
\makeatletter
\def\maxwidth{ %
  \ifdim\Gin@nat@width>\linewidth
    \linewidth
  \else
    \Gin@nat@width
  \fi
}
\makeatother

\definecolor{fgcolor}{rgb}{0.345, 0.345, 0.345}
\newcommand{\hlnum}[1]{\textcolor[rgb]{0.686,0.059,0.569}{#1}}%
\newcommand{\hlstr}[1]{\textcolor[rgb]{0.192,0.494,0.8}{#1}}%
\newcommand{\hlcom}[1]{\textcolor[rgb]{0.678,0.584,0.686}{\textit{#1}}}%
\newcommand{\hlopt}[1]{\textcolor[rgb]{0,0,0}{#1}}%
\newcommand{\hlstd}[1]{\textcolor[rgb]{0.345,0.345,0.345}{#1}}%
\newcommand{\hlkwa}[1]{\textcolor[rgb]{0.161,0.373,0.58}{\textbf{#1}}}%
\newcommand{\hlkwb}[1]{\textcolor[rgb]{0.69,0.353,0.396}{#1}}%
\newcommand{\hlkwc}[1]{\textcolor[rgb]{0.333,0.667,0.333}{#1}}%
\newcommand{\hlkwd}[1]{\textcolor[rgb]{0.737,0.353,0.396}{\textbf{#1}}}%

\usepackage{framed}
\makeatletter
\newenvironment{kframe}{%
 \def\at@end@of@kframe{}%
 \ifinner\ifhmode%
  \def\at@end@of@kframe{\end{minipage}}%
  \begin{minipage}{\columnwidth}%
 \fi\fi%
 \def\FrameCommand##1{\hskip\@totalleftmargin \hskip-\fboxsep
 \colorbox{shadecolor}{##1}\hskip-\fboxsep
     % There is no \\@totalrightmargin, so:
     \hskip-\linewidth \hskip-\@totalleftmargin \hskip\columnwidth}%
 \MakeFramed {\advance\hsize-\width
   \@totalleftmargin\z@ \linewidth\hsize
   \@setminipage}}%
 {\par\unskip\endMakeFramed%
 \at@end@of@kframe}
\makeatother

\definecolor{shadecolor}{rgb}{.97, .97, .97}
\definecolor{messagecolor}{rgb}{0, 0, 0}
\definecolor{warningcolor}{rgb}{1, 0, 1}
\definecolor{errorcolor}{rgb}{1, 0, 0}
\newenvironment{knitrout}{}{} % an empty environment to be redefined in TeX

\usepackage{alltt}

%% \title{Analysis of the {\ttfamily GSOEP9402} data}
%% \author{Arthur Allignol}
%% \date{}

\usepackage{booktabs}
\usepackage[osf]{sourcesanspro}
\usepackage{sourcecodepro}
\usepackage[T1]{fontenc}
%\renewcommand*{\ttdefault}{sourcecodepro}
\usepackage{subcaption}
\usepackage{graphicx,rotating,epsfig,multirow,multicol,hhline}
\usepackage{amsmath,amsthm,amssymb,amsfonts}
\usepackage{url}
\usepackage{hyperref}

\renewcommand{\familydefault}{\sfdefault}
\usepackage{sfmath}


\def\bea{\begin{eqnarray*}}
\def\eea{\end{eqnarray*}}
%Dense format
\setlength{\parindent}{0em} \setlength{\textwidth}{16cm} \setlength{\textheight}{25cm}
\setlength{\topmargin}{-0.5cm} \setlength{\oddsidemargin}{0cm} \setlength{\headheight}{0cm}
\setlength{\headsep}{0cm}
\newcommand{\nat}{{\it I\hspace{-0.12cm}N}}
\pagestyle{empty}


\newcommand{\p}{{\rm P}}
\newcommand{\erw}{{\rm E}}
\newcommand{\var}{{\rm var}}
\newcommand{\eins}{{\bf 1}}%{\mathbf{1}}
\newcommand{\dif}{{\rm d}}
\newcommand{\cif}{{\rm CIF}}
\newcommand{\Prob}{\mathbb{P}}
\newcommand{\R}{\mathbb{R}}
\newcommand{\D}{\mathrm{d}}

%Special commands:
\newcommand{\Bin}{\operatorname{Bin}} % Binomial Distribution
\newcommand{\NegBin}{\operatorname{NegBin}} % Negative Bin
\newcommand{\HypGeom}{\operatorname{HypGeom}} % Hypergeometric Distribution
\newcommand{\Pois}{\operatorname{Po}} % Hypergeometric Distribution
\newcommand{\Po}{\operatorname{Po}} %
\newcommand{\Exp}{\operatorname{Exp}} %
\newcommand{\Par}{\operatorname{Par}} %
\newcommand{\Ga}{\mathcal{G}a} %
\newcommand{\Be}{\mathcal{B}e} %
\newcommand{\Var}{\operatorname{Var}} %
\newcommand{\E}{\operatorname{E}} %
\newcommand{\Cov}{\operatorname{Cov}} %
\newcommand{\MSE}{\operatorname{MSE}}

\DeclareMathOperator{\Nor}{N} % Normal -
\DeclareMathOperator{\Log}{Log} % Logistische Verteilung -
\newcommand{\ml}[2][1]{% % für Maximum-Likelihood-Schätzer von #1
\ifthenelse{#1 = 1}%
 {\hat{#2}_{\scriptscriptstyle{ML}}}%
 {\hat{#2}^{#1}_{\scriptscriptstyle{ML}}}% z.B. für sigmadach^2
}

%%%%%%%%%%%%%%%%%%%%%%%%%%%%%%%%%%%%%%%%%%%%%%%%%%%%%%%%%%%%%%%%%%%%%%
% Environment for Aufgaben
%%%%%%%%%%%%%%%%%%%%%%%%%%%%%%%%%%%%%%%%%%%%%%%%%%%%%%%%%%%%%%%%%%%%%%

\newcounter{ka}


\newtheorem{exercise}{Problem}
\newenvironment{aufgabe}{\begin{exercise}\sf}{\end{exercise} \bigskip}

\newcommand{\footer}{ \vfill
  \mbox{}\hrulefill\\
  Upload your solutions until the 23.6.2016, 10h00 s.t. on
  Moodle. Only one solution per group. Please upload the source files
  as well as the solutions.}
%\xxivtime   Uhrzeit für letzte Änderung

\pagestyle{empty}

\renewcommand{\labelenumi}{(\alph{enumi})}
\IfFileExists{upquote.sty}{\usepackage{upquote}}{}
\begin{document}

\renewcommand{\baselinestretch}{1}

\hrulefill\\
{\bf WiMa-Praktikum II -- Stochastic} \hspace{\fill} Summer term 2016\\
Arthur Allignol \hspace{\fill} 16.6.2016\\[-1.2ex]
\mbox{}\hrulefill\\
\newline \renewcommand{\baselinestretch}{1}
\setcounter{ka}{0} \vspace{-0.5cm}



\begin{center}
\large{{\bf Exercise Sheet 6 --- Mixed-Effects Model}}
\end{center}

\begin{aufgabe}{{\bf The Study of Instructional Improvement Project}}

  The SII\footnote{Hill, H., Rowan, B., and Ball, D. (2005). Effect of
    teachers mathematical knowledge for teaching on student
    achievement. American Educational Research Journal, 42, 371-406.}
  was carried out to assess the math achievement scores of first and
  third-grade pupils in randomly selected classrooms from a national
  US sample of elementary schools. The data set includes results for
  1190 first-grade pupils sampled from 312 classrooms in 107 schools.

  The data set {\tt school.csv} contains the school level variables
  \begin{description}
  \item[schoolid] School id number
  \item[housepov] \% of households in the neighbourhood of the school
    below the poverty level. 
  \end{description}

  The data set {\tt class.csv} includes the classroom level variables
  \begin{description}
  \item[classid] Classroom id number
  \item[yearstea] Years of teacher's experience in teaching in the
    first grade
  \item[mathprep] Number of preparatory courses on the first-grade
    math contents and methods followed by the teacher (a numeric value
    from 1 to 6)
  \item[mathknow] Teacher's knowledge of the first-grade math content
    (the higher the better)
  \end{description}

  The data set {\tt pupil.csv} is concerned with the pupil level variables
  \begin{description}
  \item[childid] Pupil's id number
  \item[mathgain] Pupil's gain in the math achievement score from the
    spring of kindergarten to the spring of first grade
  \item[mathkind] Pupil's math score in the spring of kindergarten
  \item[sex] Gender
  \item[minority] Indicator variable for the minority status
  \end{description}
  
  The outcome of interest in contained in the variable {\tt mathgain}.

  \begin{enumerate}
  \item The variables {\tt schoolid} and {\tt classid} are also
    present in the data sets {\tt class.csv} and {\tt
      pupil.csv}. Merge the three data sets by these variables.
  \item How many nested levels is there in the data? How many random
    effects would you consider?
  \item Find a good model for the data. Explain the effect of the
    predictors on the gain math achievement score.
  \end{enumerate}

  
  
\end{aufgabe}

\end{document}
