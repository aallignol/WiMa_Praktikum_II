\documentclass[a4,11pt]{article}\usepackage[]{graphicx}\usepackage[]{color}
%% maxwidth is the original width if it is less than linewidth
%% otherwise use linewidth (to make sure the graphics do not exceed the margin)
\makeatletter
\def\maxwidth{ %
  \ifdim\Gin@nat@width>\linewidth
    \linewidth
  \else
    \Gin@nat@width
  \fi
}
\makeatother

\definecolor{fgcolor}{rgb}{0.345, 0.345, 0.345}
\newcommand{\hlnum}[1]{\textcolor[rgb]{0.686,0.059,0.569}{#1}}%
\newcommand{\hlstr}[1]{\textcolor[rgb]{0.192,0.494,0.8}{#1}}%
\newcommand{\hlcom}[1]{\textcolor[rgb]{0.678,0.584,0.686}{\textit{#1}}}%
\newcommand{\hlopt}[1]{\textcolor[rgb]{0,0,0}{#1}}%
\newcommand{\hlstd}[1]{\textcolor[rgb]{0.345,0.345,0.345}{#1}}%
\newcommand{\hlkwa}[1]{\textcolor[rgb]{0.161,0.373,0.58}{\textbf{#1}}}%
\newcommand{\hlkwb}[1]{\textcolor[rgb]{0.69,0.353,0.396}{#1}}%
\newcommand{\hlkwc}[1]{\textcolor[rgb]{0.333,0.667,0.333}{#1}}%
\newcommand{\hlkwd}[1]{\textcolor[rgb]{0.737,0.353,0.396}{\textbf{#1}}}%

\usepackage{framed}
\makeatletter
\newenvironment{kframe}{%
 \def\at@end@of@kframe{}%
 \ifinner\ifhmode%
  \def\at@end@of@kframe{\end{minipage}}%
  \begin{minipage}{\columnwidth}%
 \fi\fi%
 \def\FrameCommand##1{\hskip\@totalleftmargin \hskip-\fboxsep
 \colorbox{shadecolor}{##1}\hskip-\fboxsep
     % There is no \\@totalrightmargin, so:
     \hskip-\linewidth \hskip-\@totalleftmargin \hskip\columnwidth}%
 \MakeFramed {\advance\hsize-\width
   \@totalleftmargin\z@ \linewidth\hsize
   \@setminipage}}%
 {\par\unskip\endMakeFramed%
 \at@end@of@kframe}
\makeatother

\definecolor{shadecolor}{rgb}{.97, .97, .97}
\definecolor{messagecolor}{rgb}{0, 0, 0}
\definecolor{warningcolor}{rgb}{1, 0, 1}
\definecolor{errorcolor}{rgb}{1, 0, 0}
\newenvironment{knitrout}{}{} % an empty environment to be redefined in TeX

\usepackage{alltt}

%% \title{Analysis of the {\ttfamily GSOEP9402} data}
%% \author{Arthur Allignol}
%% \date{}

\usepackage{booktabs}
\usepackage[osf]{sourcesanspro}
\usepackage{sourcecodepro}
\usepackage[T1]{fontenc}
%\renewcommand*{\ttdefault}{sourcecodepro}
\usepackage{subcaption}
\usepackage{graphicx,rotating,epsfig,multirow,multicol,hhline}
\usepackage{amsmath,amsthm,amssymb,amsfonts}
\usepackage{url}
\usepackage{hyperref}

\renewcommand{\familydefault}{\sfdefault}
\usepackage{sfmath}


\def\bea{\begin{eqnarray*}}
\def\eea{\end{eqnarray*}}
%Dense format
\setlength{\parindent}{0em} \setlength{\textwidth}{16cm} \setlength{\textheight}{25cm}
\setlength{\topmargin}{-0.5cm} \setlength{\oddsidemargin}{0cm} \setlength{\headheight}{0cm}
\setlength{\headsep}{0cm}
\newcommand{\nat}{{\it I\hspace{-0.12cm}N}}
\pagestyle{empty}


\newcommand{\p}{{\rm P}}
\newcommand{\erw}{{\rm E}}
\newcommand{\var}{{\rm var}}
\newcommand{\eins}{{\bf 1}}%{\mathbf{1}}
\newcommand{\dif}{{\rm d}}
\newcommand{\cif}{{\rm CIF}}
\newcommand{\Prob}{\mathbb{P}}
\newcommand{\R}{\mathbb{R}}
\newcommand{\D}{\mathrm{d}}

%Special commands:
\newcommand{\Bin}{\operatorname{Bin}} % Binomial Distribution
\newcommand{\NegBin}{\operatorname{NegBin}} % Negative Bin
\newcommand{\HypGeom}{\operatorname{HypGeom}} % Hypergeometric Distribution
\newcommand{\Pois}{\operatorname{Po}} % Hypergeometric Distribution
\newcommand{\Po}{\operatorname{Po}} %
\newcommand{\Exp}{\operatorname{Exp}} %
\newcommand{\Par}{\operatorname{Par}} %
\newcommand{\Ga}{\mathcal{G}a} %
\newcommand{\Be}{\mathcal{B}e} %
\newcommand{\Var}{\operatorname{Var}} %
\newcommand{\E}{\operatorname{E}} %
\newcommand{\Cov}{\operatorname{Cov}} %
\newcommand{\MSE}{\operatorname{MSE}}

\DeclareMathOperator{\Nor}{N} % Normal -
\DeclareMathOperator{\Log}{Log} % Logistische Verteilung -
\newcommand{\ml}[2][1]{% % für Maximum-Likelihood-Schätzer von #1
\ifthenelse{#1 = 1}%
 {\hat{#2}_{\scriptscriptstyle{ML}}}%
 {\hat{#2}^{#1}_{\scriptscriptstyle{ML}}}% z.B. für sigmadach^2
}

%%%%%%%%%%%%%%%%%%%%%%%%%%%%%%%%%%%%%%%%%%%%%%%%%%%%%%%%%%%%%%%%%%%%%%
% Environment for Aufgaben
%%%%%%%%%%%%%%%%%%%%%%%%%%%%%%%%%%%%%%%%%%%%%%%%%%%%%%%%%%%%%%%%%%%%%%

\newcounter{ka}


\newtheorem{exercise}{Problem}
\newenvironment{aufgabe}{\begin{exercise}\sf}{\end{exercise} \bigskip}

\newcommand{\footer}{ \vfill
  \mbox{}\hrulefill\\
  Upload your solutions until the 16.6.2016, 10h00 s.t. on
  Moodle. Only one solution per group. Please upload the source files
  as well as the solutions.}
%\xxivtime   Uhrzeit für letzte Änderung

\pagestyle{empty}

\renewcommand{\labelenumi}{(\alph{enumi})}
\IfFileExists{upquote.sty}{\usepackage{upquote}}{}
\begin{document}

\renewcommand{\baselinestretch}{1}

\hrulefill\\
{\bf WiMa-Praktikum II -- Stochastic} \hspace{\fill} Summer term 2016\\
Arthur Allignol \hspace{\fill} 9.6.2016\\[-1.2ex]
\mbox{}\hrulefill\\
\newline \renewcommand{\baselinestretch}{1}
\setcounter{ka}{0} \vspace{-0.5cm}



\begin{center}
\large{{\bf Exercise Sheet 5 --- Generalized Linear Model}}
\end{center}

\begin{aufgabe}{{\bf Titanic Passenger Survival}}

  The data set {\tt titanic.csv} contains information on the fate of
  891 passengers of the Titanic. Additionally, we have for each
  passenger the following information
  \begin{description}
  \item[PassengerId] Passenger ID
  \item[Pclass] Passenger Class
  \item[Name] Passenger name
  \item[Sex] Gender
  \item[Age] Age
  \item[SibSp]     Number of Siblings/Spouses Aboard
  \item[Parch]     Number of Parents/Children Aboard
  \item[Ticket] Ticket number
  \item[Fare] Passenger fare
  \item[Cabin] Cabin number
  \item[Embarked] Port of embarkation (C = Cherbourg; Q = Queenstown;
    S = Southampton)
  \end{description}
  The variable {\tt Survived} contains the passenger survival
  indicator (1 for survivors). The aim of the exercise is to predict the
  passengers survival.

  \begin{enumerate}
  \item {\tt Age} has 177 missing values. Perform a crude imputation
    by replacing the missing ages by the median age computed for each {\tt
      Pclass}. Motivate this approach.
  \item Most families were given one ticket for all family members,
    whose price is recorded in variable {\tt Fare}. Divide the fare by
    the number of individuals sharing the same ticket number.
  \item Find a good model that predicts survival (Hint: You can
    compare nested models with {\tt anova(model1, model2, test =
      ``Chisq'')}. Otherwise you can look at the AIC in the {\tt
      summary} output. The AIC is a function of the log-likelihood and
    the number of parameters, thus provides a trade-off between
    complexity of the model and fit. The smaller the better. And don't
    forget simple graphical analyses.) 
  \item Based on your chosen model, predict the fate of Jack Dawson
    (Leonardo DiCaprio) and Rose DeWitt Bukater (Kate Winslet)
  \end{enumerate}
  
  
\end{aufgabe}

\end{document}
